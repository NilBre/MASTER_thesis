\chapter{Methodik des Strangetaggings}
% Das Strange-Tagging befasst sich damit, Hadronen mit Strange-Quarks
% zu identifizieren.
%
% Analog zum b-Tagging, bei welchem zum Beispiel Sto\ss parameter,
% Flugstrecke, und Zerfallsvertizes verwendet werden um Ausk\"unfte \"uber
% in den Hadronen enthaltene bottom-Quarks zu erhalten,
% kann beim Strange-Tagging, welches hier f\"ur das ungeladene
% $\symup{\Lambda}$-Baryon durchgef\"uhrt wurde, die Informationen von $\symup{\Delta R}$,
% $\symup{p_{T}(\Lambda)} / \symup{p_{T}(\text{Jet})}$ und eventuell der
% origins der Strange-Quarks als Auskunft \"uber Strange-Quarks
% verwendet werden.
%
% In diesem Fall werden $\symup{\Lambda}$-Baryonen mit dem gr\"o\ss ten
% Verzweigungsverhältnis mit zwei Tochterteilchen rekonstruiert.
% Mit den obigen Diskriminanten k\"onnen Aussagen \"uber die Signaleffizienz
% getroffen werden.
