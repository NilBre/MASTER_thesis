\chapter*{Abstract}
\label{sec:abstract}

% was ist strange tagging und wozu braucht man es? -> check
% was sind meine hauptergebnisse? -> check

% In dieser Bachelorarbeit wird ein Algorithmus zur Rekonstruktion
% von Λ-Baryonen implementiert, welcher auf bereits bestehenden
% Algorithmen zur Rekonstruktion von Kaonen aufbaut und zur Verbesserung
% von Strange-Tagging beiträgt.
% Strange-Tagging ist eine Form des Flavortaggings, welche darauf
% ausgelegt ist, Hadronen mit Strange-Quarks zu identifizieren.
% Vor allem f\"ur neutral geladene Hadronen, welche mindestens ein
% Strange-Quark als Parton tragen ist es interessant
% herauszufinden in welchen Prozessen diese entstehen.
% Au\ss erdem kann das Strange-Tagging dabei helfen, neue Erkenntnisse
% \"uber das CKM-Matrix Element $|\symup{V}_{ts}|$ zu erzielen.
% Dafür werden verschiedene Objekte herangezogen, welche zwischen
% Strange-Quark-Jets und Up-Quark-Jets beziehungsweise Down-Quark-Jets
% diskriminieren.
% Es ergibt sich, dass die Objekte $X_{\symup{\Lambda}}$ und
% $\symup{\Delta R}$ gut zwischen Strange-Jets von Down- und Up-Jets
% diskriminieren doch die Masse des $\symup{\Lambda}$-Baryons daf\"ur
% nicht geeignet ist.
