\chapter*{Abstract}
\label{sec:abstract}

Due to larger luminosities and higher track multiplicity, detectors with finer granularity are needed.
The LHCb experiment is undergoing a major during the long shutdown of the LHCb between 2019 and 2022. The three tracking detectors downstream of the dipole magnets are undergoing a replacement with a detector that consists of scintillating fibers (SciFi).
The calibration of the new detector to the software regarding
orientation and position is critical for the subsequent performance. This
process is called \textit{alignment} and is part of the trigger and is a crucial part of the real-time analysis of the LHCb.
\\
In this work, the software \textit{alignment} of the SciFi tracker is studied.
So called \textit{misalignment} tests contributed to determine the quality
of the alignment. In addition, tests for the identification of weak modes were performed.
This drew attention to a bias within the SciFi hit clusters which was observed to have an impact on the performance of the alignment procedure.

% more precise and give results!

\chapter*{Kurzfassung}
\label{sec:kurzf}

Aufgrund größerer Luminositäten und höherer Spurmultiplizitäten werden Detektoren mit feinerer Granularität benötigt.
Das LHCb-Experiment wird während der langen Abschaltung des LHCb zwischen 2019 und 2022 einem größeren Umbau unterzogen. Die drei Spurdetektoren hinter den Dipolmagneten werden durch einen Detektor ersetzt, der aus szintillierenden Fasern (SciFi) besteht.
Die Kalibrierung des neuen Detektors mit der Software hinsichtlich
Ausrichtung und Position ist entscheidend für die spätere Leistung. Dieser
Prozess wird \textit{alignment} genannt und ist Teil des Triggers und ein entscheidender Teil der Echtzeitanalyse des LHCb.
\\
In dieser Arbeit wird das Software-\textit{Alignment} des SciFi-Trackers studiert.
Sogenannte \textit{misalignment}-Tests trugen zur Bestimmung der Qualität
des Alignments. Darüber hinaus wurden Tests zur Identifizierung von schwachen Moden durchgeführt.
Dies machte auf einen Bias innerhalb der Cluster aufmerksam welcher einen
nicht-unwichtigen Einfluss auf das Alignment hat.
