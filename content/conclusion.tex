\chapter{Zusammenfassung}
% In dieser Arbeit wurde die Rekonstruktion von
% $\symup{\Lambda}$-Baryonen behandelt und ein Algorithmus implementiert,
% welcher mithilfe von verschiedenen Diskriminanten Aussagen
% dar\"uber treffen soll, wie das Strange-Tagging verbessert werden
% kann.
% Es ergibt sich aus den ROC-Kurven, dass die Objekte $\symup{\Delta R}$
% und der Quotient aus $\symup{p}_{\symup{T}}$ des $\symup{\Lambda}$-Baryons und
% Jet-Transversalimpuls $\symup{X}_{\Lambda}$, gute Diskriminanten sind, um
% $\symup{\Lambda}$-Baryonen die aus s-Jets kommen von denen aus d-Jets und u-Jets
% zu unterscheiden.
% Die Masse von $\symup{\Lambda}$-Baryonen ist daf\"ur nicht geeignet.
% Bez\"uglich $\symup{X}_{\Lambda}$ k\"onnen die Wertepaare der Werte in
% Tabelle \ref{tab:sebe} der Signaleffizienz und der Hintergrundeffizienz
% aus Abbildung \ref{fig:roc_sd} zu
% \begin{table}[H]
%   \centering
%   \begin{tabular}{c c}
%     \toprule
%     $\text{SE}$ & $\text{BE}$ \\
%     \midrule
%     \num{0.93} & \num{0.79} \\
%     \num{0.83} & \num{0.61} \\
%     \num{0.74} & \num{0.53} \\
%     \bottomrule
%   \end{tabular}
%   \caption{Wertepaare der SE und BE bez\"uglich der ROC-Kurve
%   f\"ur $\symup{X}_{\symup{\Lambda}}$ der Abbildung
%   \ref{fig:roc_sd}.}
%   \label{tab:sebe}
% \end{table}
% bestimmt werden.
% Diese Erkenntnis kann f\"ur das Strange-Tagging verwendet werden um mit
% relativ guter Pr\"azision Ausk\"unfte \"uber das Auftreten von
% Strange-Quarks in Jets zu geben.
% In auf diese Arbeit aufbauenden Analysen k\"onnte die vorangegangene
% Rekonstruktion von $\symup{K}_{S}$-Mesonen mit meinem Algorithmus kombiniert werden, in der Hoffnung noch h\"ohere
% Taggingeffizienzen zu erhalten.
