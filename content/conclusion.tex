\chapter{Conclusion}

Starting with the Nulltests it was found that constraining $Tx$ and $Tz$ in the last C-frame in station 3 combined with the rotation around $z$-axis yields the best results in terms of bringing down z-rotation and improving the overall alignment of the SciFi Tracker.
Only constraining station 3 in the given translatory and rotational degrees of freedom improved the alignment for the first two stations regarding $Tz$.

The tests of the different track selections showed a strong correlation between the $\chi^2 / \text{dofs}$ and the number of tracks.
For the GoodLongTracks, the additional alignment of the modules was found to improve the $\chi^2 / \text{dofs}$ as well as the use of two translational and two rotational degrees of freedom compared to only Tx and Rz.
The spikey behavior seen in the $\chi^2 / \text{dofs}$ versus iteration number measurement for the \textit{HighMomentumTTracks} still has unknown sources.

The measurements comparing maximum luminosity to ramp-up luminosity show small differences in $Tx$, $Rz$, and $\chi^2 / \text{dofs}$. Furthermore, a fast convergence can be observed which is either due to the absence of weak modes in the data set or to the optimized configuration.

Furthermore, a cluster bias was detected which was fixed with a temporary correction. The impact of the cluster bias is clearly visible in $Rz$ and will make a significant difference during alignment once it is fixed.

In addition, misalignment tests were performed to determine if the SciFi detector is capable of correcting the misaligned condition. Eight $\SI{100}{\micro\metre}$ translation misalignment samples were tested to exclude further biases. From the convergence it can be seen that the SciFi detector returns to the aligned state in all of the eight cases.
