\chapter{Conclusion}

\chapter{Conclusion}

In order to handle increasing instantaneous luminosity and read out the data at $\SI{40}{\mega\hertz}$, the LHCb has been upgraded. The new SciFi tracker is currently being commissioned, and the physics performance at LHCb will depend on how well the SciFi is aligned since poor alignment leads to systematic biases which can have a negative impact on sensitive asymmetry measurements. This thesis describes tests for the alignment using simulated data, which are used to find the best configuration for aligning the real detector and measure the expected accuracy of alignment.

Null tests were used to search for and correct for weak modes in the reconstruction software. A configuration called ``config5 Rz'', was found which gave the best results in this test. This constrains $Tx$, $Tz$ and $Rz$ in the last C-frame in station three.
Only constraining station three in the given translational and rotational degrees of freedom improved the alignment for the first two stations regarding $Tz$.

Possible alignment configurations were also tested with input misalignment to the detector. After misaligning the modules in $Tx$, $Ty$ and $Tz$ for every layer in each station by $\SI{0.001}{\micro\metre}$ or by $\SI{200}{\micro\metre}$, the system can return to its original alignment with differences that are small enough to be called negligible. The accuracy of the SciFi alignment to realistic translations is very good. On the other hand, it was found that misaligning the Modules in $Rx$, $Ry$ and $Rz$ for every layer in each station by $\SI{0.01}{\milli\radian}$ resulted in misalignment of the SciFi detector that was difficult to correct with the alignment system, especially seen in $Tz$ and $Rz$. The noticeable misalignment in $Rz$ is also enlarged by the cluster bias effect. The cluster bias has a negative impact on $Rz$ and increases the rotation around $z$ by a factor of 2 compared to the simulation using a temporary fix for the bias. The true fix is currently being monitored and will improve the alignment once a solution is implemented.

Finally, the track selections for the alignment were also tested as part of the alignment configuration. The tests of the different track selections showed a strong correlation between the $\chi^2 / \text{dofs}$ and the number of tracks.
For the GoodLongTracks, including the additional alignment of the modules was found to improve the $\chi^2 / \text{dofs}$. The use of two translational and two rotational degrees of freedom compared to only $Tx$ and $Rz$ also improves the $\chi^2 / \text{dofs}$ and the alignment.

The alignables, track types and degrees of freedom used in the last $\chi^2 / \text{dof}$ tests will be used for the upcoming misalignment tests when the cluster bias is fixed and when the alignment with real data will begin.

% finished
