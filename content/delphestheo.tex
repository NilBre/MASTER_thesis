\subsection{DELPHES \cite{delph}}
% DELPHES Version 3.4.1 ist ein Simulationsprogramm für radialsymmetrische
% Teilchendetektoren. Hier wird ein Detektor,
% welcher dem CMS-Detektor
% \"ahnelt, verwendet als Beispiel f\"ur einen Multifunktionsdetektor des LHC
% \cite{jerdmannTagger}.
% Es simuliert einen Spurdetektor(Tracker)
% im Inneren eines Magnetfeldes, ein elektromagnetisches(ECAL) und
% ein hadronisches Kalorimeter(HCAL) und außerdem einen Myon-Detektor.
% Es können so physikalische Objekte durch die simulierten
% Detektorantworten rekonstruiert werden. Dafür stehen
% Kalorimetereinträge, Informationen zu Isolationskriterien für
% Elektronen, Tau-Leptonen, sowie Jets und Informationen zur
% fehlenden Energie zur Verfügung. Zur Funktionsweise der einzelnen
% Detektoren, im Spurdetektor werden Teilchenpropagationen betrachtet.
% Dafür wird zwischen geladenen und ungeladenen Teilchen unterschieden.
% Ungeladene Teilchen folgen einer linearen Trajektorie, wohingegen
% geladene Teilchen einer gekrümmten Bahn, abhängig vom Magnetfeld
% folgen. Außerhalb des Trackers startende Teilchenspuren
% werden ignoriert. Für geladene Teilchen kann der Benutzer
% eine Rekonstruktionswahrscheinlichkeit einstellen, mit
% welcher ein Teilchen als Spur im Tracker rekonstruiert
% wird. Die Winkelauflösung sei dabei exakt und der
% Transversalimpuls gemäß einer Gaußkurve verschmiert. Die
% Kalorimeter verschmieren die Energieanteile $f_{\text{ECAL}}$ und $f_{\text{HCAL}}$
% unkorreliert voneinander.
%
% Hinsichtlich der Kalorimeter ist das ECAL für die Energiebestimmung von
% Elektronen und Photonen zuständig und das HCAL für (un-)geladene Hadronen.
% Dies gilt nur für "langlebige" Hadronen. Hierzu zählen $\symup{\Lambda}$-Baryonen und
% Kaonen. Obwohl sie nur eine endliche Lebensdauer haben, gelten sie als stabil.
% Da sie einen nicht zu vernachläßigbaren Anteil ihrer Energie auch im
% ECAL deponieren würden, werden ihre Energiedepositionsanteile zu
% $f_{ECAL} = 0.3$ und $f_{HCAL} = 0.7$ definiert.
% Elektronen und Photonen deponieren ihre gesamte Energie im
% elektromagnetischen Kalorimeter.
%
% In der Objektrekonstruktion werden diverse Annahmen bzgl. der
% Teilchenarten getroffen. Zu den geladenen Leptonen zählen nur
% Elektronen und Myonen, da die Tau-Leptonen zu schnell zerfallen.
% Photonen werden so rekonstruiert, dass Elektronen und Photonen ohne Spur als
% Photon gewertet werden.
% Außerdem kann ein Teilchen isoliert sein. Das ist dann der Fall, wenn sein
% Abstand zu anderen Teilchen kleiner ist als ein definierter Abstand L.
% Ein sehr wichtiger Teil der Objektrekonstruktion ist die der Jets, da diese die
% häufigsten Endzustände bilden.
% DELPHES bietet drei große Jetklassen, welche sich in ihrem Input unterscheiden.
% Zum einen die \textit{Generated Jets}, welche aus langlebigen Teilchen nach
% den Parton-Showern und Hadronisierungsprozessen entstehen. Hierbei werden
% keine Detektorsimulations- bzw. Rekonstruktionsinformationen beachtet.
% Zum Zweiten die \textit{Calorimeter Jets}, welche Kalorimetereinträge als
% Eingabe erhalten und zum Dritten die \textit{particle flow jets}, welche das
% Resultat von Zusammenschlüssen von particle-flow Spuren sind.
% Abschließend werden Informationen über fehlende Transversalenergie
% $\symup{\vec{E}_{T}^{\text{miss}}}$ zurate gezogen, um Aussagen über Teilchen wie die
% Neutrinos zu treffen.
% Dabei ist die fehlende Transversalenergie als
% \begin{equation}
%   \symup{\vec{E}_{T}^{\text{miss}}} = -\sum_{i} \vec{p}_{T}(i)
% \end{equation}
% definiert.
