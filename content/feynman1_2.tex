%FEYNMAN-Diagramm f\"ur die Erzeugung des ersten Datensatzes.
\begin{figure}[H]
\centering
\begin{subfigure}{0.3\textwidth}
\feynmandiagram [horizontal=a to b]{
i1 --[gluon, edge label=\(g\)] a --[gluon, edge label=\(g\)] i2,
a --[gluon, edge label=\(g\)] b,
f1 --[fermion, edge label=\(\bar{s}\)] b --[fermion, edge label=\(s\)] f2,
};
\caption{\label{fig:feyn1}}
\end{subfigure}
\begin{subfigure}{0.3\textwidth}
\feynmandiagram [horizontal=a to b]{
i1 --[gluon, edge label=\(g\)] a --[gluon, edge label=\(g\)] i2,
a --[gluon, edge label=\(g\)] b,
f1 --[fermion, edge label=\(\bar{d}\)] b --[fermion, edge label=\(d\)] f2,
};
\caption{\label{fig:feyn2}}
\end{subfigure}
\begin{subfigure}{0.3\textwidth}
\feynmandiagram [horizontal=a to b]{
i1 --[gluon, edge label=\(g\)] a --[gluon, edge label=\(g\)] i2,
a --[gluon, edge label=\(g\)] b,
f1 --[fermion, edge label=\(\bar{u}\)] b --[fermion, edge label=\(u\)] f2,
};
\caption{\label{fig:feyn3}}
\end{subfigure}
\caption{FEYNMAN-Diagramme f\"ur die Erzeugung der Datens\"atze.}
\label{fig:allFeyn}
\end{figure}
