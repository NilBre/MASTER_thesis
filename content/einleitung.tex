\chapter{Introduction}
\label{sec:einleitung}
% Reihenfolge:\\
% um was geht es im grossen ganzen? wozu braucht man teilchenidentifikation und Strange tagging?\\
% was ist die grundidee des Strange-Taggings? (meinen absatz da einf\"ugen)\\
% dann eventuell b-tagging erkl\"aren\\
% was ist das ziel meiner studie und wie gehe ich vor? (kurze zusammenfassung meiner struktur hilfreich)\\

% Die Rekonstruktion von Teilchen hat in der Teilchenphysik eine hohe Priorität,
% da die Analyse dieser nur indirekt erfolgen kann. Deshalb wird auf
% Programme zurückgegriffen, welche teilchenphysikalische Prozesse zunächst
% simulieren sollen, um ein besseres Verständnis dafür zu bekommen, worauf
% das Augenmerk gelegt werden soll.
% Schaut man nun ein Ereignis in einem Detektor wie ATLAS an
% wird schnell klar, dass viele Programme zur Teilchenidentifikation
% notwendig sind, um die resultierenden Spuren auszuwerten.
%
% Der ATLAS Detektor ist ein radialer Detektor und besteht aus
% mehreren Schichten. In der Mitte liegt das Strahlrohr, um welches der Spurdetektor gebaut ist. Ein starkes Magnetfeld befindet sich im gesamten Spurdetektor. Dieser dient dazu, die Spuren geladener Teilchen zu
% rekonstruieren.
% Anschlie\ss end findet man ein elektromagnetisches Kalorimeter, worin
% haupts\"achlich langlebige Leptonen und Photonen ihre Energie
% deponieren. Dahinter befindet sich das hadronische Kalorimeter, in
% welchem Hadronen ihren Hauptenergiebetrag deponieren. In der
% \"au\ss ersten Schicht des Torus und in den Endkappen befinden
% sich die sogenannten Myonenkammern, welche allein zur
% Myondetektion beitragen.
%
% Wenn neutral geladene Hadronen wie das $\symup{\Lambda}$-Baryon
% die Detektorschichten passieren, ist es schwierig diese
% zu identifizieren.
%
% Mit Hilfe von Ereignisgeneratoren wie Madgraph, Showerprogrammen wie Pythia und
% Detektorsimulatoren wie DELPHES können künstlich Teilchenkollisionen
% simuliert werden.
% Um ein bestimmtes Teilchen rekonstruieren zu wollen, müssen zunächst
% bestimmte Kriterien betrachtet werden, das bedeutet es muss gegeben
% sein, dass dieses Teilchen künstlich generiert werden kann und die
% Verzweigungsverhältnisse nicht zu klein sind.
% Hier wird die Rekonstruktion von
% $\symup{\Lambda}$-Baryonen untersucht. Das bedeutet, dass
% die Schwierigkeit darin besteht, das Strange-Quark zu kennzeichnen.
% Das f\"uhrt auf die sogenannten Strange-Tagger.
% Strange-Tagging Algorithmen helfen dort aus, schwer zu identifizierende
% Strange-Hadronen, aufgrund kleiner Verzweigungsverhältnisse,
% besser zu untersuchen.
%
% Das Strange-Tagging befasst sich damit, Hadronen mit Strange-Quarks
% zu identifizieren.
% Hier werden dazu verschiedene Prozesse betrachtet, bei welchen
% $\symup{\Lambda}$-Baryonen im Endzustand herauskommen.
% Hier sind das die Prozesse $pp \rightarrow s\bar{s}$, $pp \rightarrow d\bar{d}$
% und $pp \rightarrow u\bar{u}$, welche durch den MadGraph Ereignisgenerator erzeugt werden.
% Durch Betrachtung verschiedener Objekte wie $X_{\Lambda}$ und
% $\symup{\Delta R}$ kann mittels ROC-Kurven(receiver operating characteristic) Auswertung eine
% Aussage \"uber die Signaleffizienz der $\symup{\Lambda}$ aus diesen Prozessen
% getroffen werden.
% Hier wird nur der Fall betrachtet, in welchem $\symup{\Lambda}$-Baryonen
% mit dem gr\"o\ss ten Verzweigungsverhältnis in zwei
% Tochterteilchen zerfallen und dessen Tochterteilchen geladene
% Hadronen sind.
%
% Der Anreiz, solche Methoden und Techniken weiterzustudieren, kommt
% von der Methodik des b-Taggings zum Nachweis von bottom-Quarks
% in B-Hadronen.
%
% Beim b-tagging \cite{btag1} wird verwendet, dass Jets, welche bottom-Quarks enthalten,
% im Vergleich zu anderen Jets eine verhältnismäßig lange Lebensdauer
% von $\SI{1.5}{\pico\second}$ haben. Damit ist die Flugstrecke ein
% paar Millimeter lang bevor sie in leichtere Hadronen zerfallen.
% Die zwei hauptsächlich verwendeten \textit{Tagger} sind zum einen
% der \textit{spatial Tagger} und zum anderen der \textit{soft lepton Tagger}.
% Beim \textit{spatial Tagger} werden Informationen über die Lebensspanne
% der Teilchen, wie die Flugstrecke, Stoßparameter und Zerfallsvertizes
% verwendet. Hierzu werden die Sekundär-Vertizes verwendet, um Aufschlüsse
% über den Zerfall zu gewinnen.
% Bei dem \textit{soft lepton Tagger} wird die
% Rekonstruktion der Leptonen im semileptonischen Zerfall verwendet.
%
% Abzielend auf die Rekonstruktion von $\symup{\Lambda}$-Baryonen beginne
% ich mit einem Konsistenztest, indem ich die Verzweigungsverhältnisse des
% betrachteten $\symup{\Lambda}$-Zerfalls und des $K_{S}$ Zerfalls nach
% $\pi^{+}\pi^{-}$, f\"ur welches bereits ein Rekonstruktionsalgorithmus
% existiert, aus meinen Samples nachrechne.
% Dies soll Gewissheit schaffen, dass die Samples konsistente Ergebnisse
% mit dem PDG liefern.
% Anschlie\ss end wird der bereits vorhandene Algorithmus f\"ur
% die Rekonstruktion der $K_{S}$ ausgewertet, indem Objekte und ROC-Kurven
% interpretiert und ausgewertet werden.
% Im Anschluss werden die Ergebnisse des implementierten Algorithmus
% der $\symup{\Lambda}$-Rekonstruktion ausgewertet und analysiert.
% Dies beinhaltet Unterschiede zwischen den Plots herauszulesen und die ROC-Kurven zu interpretieren.
