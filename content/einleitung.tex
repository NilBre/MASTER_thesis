\chapter{Introduction}
\label{sec:einleitung}

At the beginning of the $20^{\text{th}}$ century many physicists started research on
elementary particles and the interactions associated with them. The combined
knowledge lead to the construction of one of the most precisely tested theories: the Standard Model (SM) of particle physics.
Flavour anomalies show strong tensions with the Standard Model and also the recent publication on the W-boson mass measurement is challenging the accuracy of the Standard Model\cite{wmass}.
No single measurement or anomaly on its own is enough to be in a total disagreement with the SM, but combined they provide hints that the SM is not the final theory.
The SM describes every fundamental force except for gravity. There are still open questions such the baryon asymmetry of the universe requiring a, by several orders of magnitude, larger charge-parity (CP) violation than the SM predicted.
To tackle these problems, high energy experiments such as the Large Hadron Collider beauty (LHCb) experiment located at the Large Hadron Collider (LHC) at CERN were built for this exact reason.
The LHCb experiment was designed to study beauty and charm quarks with focus on measuring CP-violation and searching for New Physics in rare decays.
To detect these phenomena, the threshold for statistical uncertainties has to
be lowered and the amount of data collected needs to be increased. The LHCb upgrade described in section \ref{sec:upgradeLHCb} allows us to have a five times higher instantaneous luminosity of $\SI{2e33}{\invfb}$ with the expected detector readout rate of $\SI{40}{\mega\hertz}$.
To realize these hardware and software challenges, the front-end electronics, tracking systems and the trigger system needed upgrades.
The tracking system has been replaced with a single tracker based on scintillating fibres that is currently commissioned. The physics performance is highly dependent on how well the detector is aligned, since poor alignment leads to systematic biases which can have a negative impact on sensitive asymmetry measurements. It can also lead to worse mass resolution. Therefore it is crucial that the SciFi detector is well aligned.
To operate the upgraded LHCb at its full potential, it is of great importance that all detector components are brought into an alignment level so that the impact on physical observables is insignificant.
\\
\\
The Alignment theory will be described in chapter \ref{sec:alignTheory}. In chapter \ref{sec:story} different sets of constraints, degrees of freedom
and alignable objects called \textit{configuration} will be tested first in order
to study how different configurations influence the alignment. Afterwards several
tests will be performed to analyse the behavior of a misaligned detector and check
if the chosen configuration converges towards an aligned state.
The deviations from a centered state after the alignment are needed for other trigger stages to be known in the reconstruction.
The LHC will not run permanently at maximum luminosity therefore tests are performed to analyse alignment of different luminosity samples. Especially while the LHC restarts it will run at a lower luminosity.
During the alignment studies a bias inside the Scintillating Fibre hit clustering algorithms was discovered which had an impact on the alignment. The exact changes will be discussed in the final section of chapter \ref{sec:story}.

% finished
