\chapter{Introduction}
\label{sec:einleitung}

The beginning of the $20^{\text{th}}$ century many physicists started research on
elementary particles and the interactions associated with them. The combined
knowledge lead to the construction of most precisely tested theories: the
Standard Model (SM) of particles. To this day the measurements are in agreement
with the model within the theretical and experimental uncertainties.
The SM describes every fundamental force except for gravity. There are still open
questions such the matter-antimatter asymmetry in the universe leading to a larger
charge-parity (CP) violation than the SM predicted.
To takle these problems, high energy experiments such as the LHCb located at the
Large Hadron Collider (LHC) at CERN were build for this exact reason.
The LHCb's main purposes are beauty-quark and charm-quark physics especially
precision measurements with a focus and decays involving thes quarks.
These measurements provide insights into CP-violation for massive
particle decays.
To detect these phenomena the threshold for statistical uncertainties has to
be lowered and the amount of data collected needs to be increased. The upgrade
described in section \ref{sec:upgradeLHCb} will increase the instantaneous
luminosity by a factor of five to $\num{2e33}$ $\symup{fb}^{-1}$ and
the detector readout rate will be at $\SI{40}{\mega\hertz}$. To realize these
hardware challenges the frontend electronics and tracking systems needed upgrades.
The operate the upgraded Tracker at its full potential the software must be
calibrated as good as possible to the physical detector.
\\
\\
The Alignment theory will be described in \ref{sec:alignTheory}.
In chapter \label{sec:story} different sets of constraints, degrees of freedom
and alignable objects called \textit{configuration} will be tested first in order
to study how different configurations influence the alignment. Afterwards several
tests will be performed to analyse the behavior of a misaligned detector and check
if the chosen configuration converges towards an aligned state. Since the detector
cannot be aligned the whole time, it is crucial that misalignments do not stay
permanent but can be reverted.
The LHC will not run permanently at maximum luminosity therefore tests are performed
to analyse alignment of different luminosity samples.
During the alignment studies a bias inside the clustering algorithms was discovered
which had an impact of the alignment. The exact changes will be discussed in the final section of chapter \ref{sec:story}.
