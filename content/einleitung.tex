\chapter{Introduction}
\label{sec:einleitung}

At the beginning of the $20^{\text{th}}$ century many physicists started research on
elementary particles and the interactions associated with them. The combined
knowledge lead to the construction of most precisely tested theories: the
Standard Model (SM) of particles.
Flavor anomalies show strong tensions with Standard Model and also the recent publication on the W-boson mass calculation is challenging the accuracy of the Standard Model. Despite these phenomena the measurements are in agreement
with the model within the theoretical and experimental uncertainties.
The SM describes every fundamental force except for gravity. There are still open
questions such as the baryon asymmetry of the universe leading to a larger
charge-parity (CP) violation than the SM predicted.
To tackle these problems, high energy experiments such as the LHCb experiment located at the Large Hadron Collider (LHC) at CERN were built for this exact reason.
The LHCb experiment was designed to study beauty and charm quarks with focus on measuring CP-violation and searching for New Physics in rare decays.
To detect these phenomena the threshold for statistical uncertainties has to
be lowered and the amount of data collected needs to be increased. The upgrade
described in section \ref{sec:upgradeLHCb} will increase the instantaneous
luminosity by a factor of five to $\SI{2e33}{\invfb}$ and
the detector readout rate will be at $\SI{40}{\mega\hertz}$. To realize these
hardware and software challenges, the front-end electronics and tracking systems needed upgrades.
To operate the upgraded LHCb experiment at its full potential the software must be
calibrated as well as possible to the physical detector.
\\
\\
The Alignment theory will be described in chapter \ref{sec:alignTheory}.
In chapter \label{sec:story} different sets of constraints, degrees of freedom
and alignable objects called \textit{configuration} will be tested first in order
to study how different configurations influence the alignment. Afterwards several
tests will be performed to analyse the behavior of a misaligned detector and check
if the chosen configuration converges towards an aligned state.

% Since the detector
% cannot be aligned the whole time, it is crucial that misalignments do not stay
% permanent but can be reverted.

The LHC will not run permanently at maximum luminosity. During the restart of the LHC it will run at lower luminosities. Therefore tests are performed to analyse alignment of different luminosity samples. During the alignment studies a bias inside the SciFi hit clustering algorithms was discovered which had an impact on the alignment. The exact changes will be discussed in the final section of chapter \ref{sec:story}.
