\chapter*{Kurzfassung}
\label{sec:kurzf}

Der LHCb-Detektor erhält ein Upgrade welches in 2019 in Form eines Long
Shutdowns des LHC begann. Die drei Spurdetektoren hinter den Dipolmagneten wurden
duch einen Detektor ausgetauscht, welcher szintillierende Fasern verwendet (SciFi).
Dies ist nur ein Teil des Upgrades. Aufgrund von größeren Luminositäten und der
steigenden Anzahl an Spurmuliplizitäten werden Detektoren mit feinerer
Granularität benötigt. Die Kalibrierung des neuen Detektors mit der Software in
Orientierung und Position ist entscheidend für die spätere Leistung. Dieser
Vorgang heißt \textit{Alignment}.
\\
In dieser Arbeit wird das Software-\textit{Alignment} des SciFi-Trackers studiert.
Durch Tests verschiedener Parameter Konfigurationen konnte ein gutes
Alignment erreicht werden. Sogenannte \textit{Misalignment}-Tests trugen dazu
bei die Qualität des Alignments zu bestimmen. Außerdem wurden Tests zur
Identifikation von schwachen Moden durchgeführt.
Dies machte auf einen Bias innerhalb der Cluster aufmerksam welcher einen
nicht-unwichtigen Einfluß auf das Alignment hat.
