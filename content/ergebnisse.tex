\chapter{Rekonstruktion von Λ-Baryonen}
\label{sec:ergebnisse}

\section{Monte Carlo Programme}
\subsection{MADGRAPH\cite{maddy}}
% Die simulierten Samples wurden mit dem Ereignisgenerator
% MADGRAPH\_MC@NLO in f\"uhrender Ordnung von
% $\symup{\alpha}_{S}$ bei einer Schwerpunktsenergie von
% $\sqrt{s} = \SI{13}{\tera\electronvolt}$ generiert.
% Die erzeugten Prozesse sind $pp \rightarrow s\bar{s}$,
% $pp \rightarrow d\bar{d}$ and $pp \rightarrow u\bar{u}$.
% Es wurden jeweils $\num{1000000}$ Ereignisse pro Prozess erzeugt.
% MADGRAPH ist das Standardwerkzeug f\"ur die Erzeugung von Matrixelementen in der Hochenergiephysik.
% Die FEYNMAN-Graphen der Prozesse sind in den Abbildungen
% \ref{fig:feyn1}, \ref{fig:feyn2} und \ref{fig:feyn3} referenziert.
%
% %FEYNMAN-Diagramm f\"ur die Erzeugung des ersten Datensatzes.
\begin{figure}[H]
\centering
\begin{subfigure}{0.3\textwidth}
\feynmandiagram [horizontal=a to b]{
i1 --[gluon, edge label=\(g\)] a --[gluon, edge label=\(g\)] i2,
a --[gluon, edge label=\(g\)] b,
f1 --[fermion, edge label=\(\bar{s}\)] b --[fermion, edge label=\(s\)] f2,
};
\caption{\label{fig:feyn1}}
\end{subfigure}
\begin{subfigure}{0.3\textwidth}
\feynmandiagram [horizontal=a to b]{
i1 --[gluon, edge label=\(g\)] a --[gluon, edge label=\(g\)] i2,
a --[gluon, edge label=\(g\)] b,
f1 --[fermion, edge label=\(\bar{d}\)] b --[fermion, edge label=\(d\)] f2,
};
\caption{\label{fig:feyn2}}
\end{subfigure}
\begin{subfigure}{0.3\textwidth}
\feynmandiagram [horizontal=a to b]{
i1 --[gluon, edge label=\(g\)] a --[gluon, edge label=\(g\)] i2,
a --[gluon, edge label=\(g\)] b,
f1 --[fermion, edge label=\(\bar{u}\)] b --[fermion, edge label=\(u\)] f2,
};
\caption{\label{fig:feyn3}}
\end{subfigure}
\caption{FEYNMAN-Diagramme f\"ur die Erzeugung der Datens\"atze.}
\label{fig:allFeyn}
\end{figure}


\subsection{PYTHIA\cite{pythia82}}
% PYTHIA ist ein Werkzeug, welches verwendet wird, um Kollisionen
% im Hochenergiebereich zu erzeugen. Au\ss erdem wird es benutzt um
% verschiedene physikalische Modelle hin zu komplexen Vielteilchen
% Endzust\"anden zu generieren.
% PYTHIA wird meist als \textit{Showerprogramm} verwendet und
% verf\"ugt \"uber Bibliotheken zu Anfangs- bzw. Endzustands
% Partonshowern, Teilchenzerf\"allen und vielem mehr.
%
% \subsection{DELPHES \cite{delph}}
% DELPHES Version 3.4.1 ist ein Simulationsprogramm für radialsymmetrische
% Teilchendetektoren. Hier wird ein Detektor,
% welcher dem CMS-Detektor
% \"ahnelt, verwendet als Beispiel f\"ur einen Multifunktionsdetektor des LHC
% \cite{jerdmannTagger}.
% Es simuliert einen Spurdetektor(Tracker)
% im Inneren eines Magnetfeldes, ein elektromagnetisches(ECAL) und
% ein hadronisches Kalorimeter(HCAL) und außerdem einen Myon-Detektor.
% Es können so physikalische Objekte durch die simulierten
% Detektorantworten rekonstruiert werden. Dafür stehen
% Kalorimetereinträge, Informationen zu Isolationskriterien für
% Elektronen, Tau-Leptonen, sowie Jets und Informationen zur
% fehlenden Energie zur Verfügung. Zur Funktionsweise der einzelnen
% Detektoren, im Spurdetektor werden Teilchenpropagationen betrachtet.
% Dafür wird zwischen geladenen und ungeladenen Teilchen unterschieden.
% Ungeladene Teilchen folgen einer linearen Trajektorie, wohingegen
% geladene Teilchen einer gekrümmten Bahn, abhängig vom Magnetfeld
% folgen. Außerhalb des Trackers startende Teilchenspuren
% werden ignoriert. Für geladene Teilchen kann der Benutzer
% eine Rekonstruktionswahrscheinlichkeit einstellen, mit
% welcher ein Teilchen als Spur im Tracker rekonstruiert
% wird. Die Winkelauflösung sei dabei exakt und der
% Transversalimpuls gemäß einer Gaußkurve verschmiert. Die
% Kalorimeter verschmieren die Energieanteile $f_{\text{ECAL}}$ und $f_{\text{HCAL}}$
% unkorreliert voneinander.
%
% Hinsichtlich der Kalorimeter ist das ECAL für die Energiebestimmung von
% Elektronen und Photonen zuständig und das HCAL für (un-)geladene Hadronen.
% Dies gilt nur für "langlebige" Hadronen. Hierzu zählen $\symup{\Lambda}$-Baryonen und
% Kaonen. Obwohl sie nur eine endliche Lebensdauer haben, gelten sie als stabil.
% Da sie einen nicht zu vernachläßigbaren Anteil ihrer Energie auch im
% ECAL deponieren würden, werden ihre Energiedepositionsanteile zu
% $f_{ECAL} = 0.3$ und $f_{HCAL} = 0.7$ definiert.
% Elektronen und Photonen deponieren ihre gesamte Energie im
% elektromagnetischen Kalorimeter.
%
% In der Objektrekonstruktion werden diverse Annahmen bzgl. der
% Teilchenarten getroffen. Zu den geladenen Leptonen zählen nur
% Elektronen und Myonen, da die Tau-Leptonen zu schnell zerfallen.
% Photonen werden so rekonstruiert, dass Elektronen und Photonen ohne Spur als
% Photon gewertet werden.
% Außerdem kann ein Teilchen isoliert sein. Das ist dann der Fall, wenn sein
% Abstand zu anderen Teilchen kleiner ist als ein definierter Abstand L.
% Ein sehr wichtiger Teil der Objektrekonstruktion ist die der Jets, da diese die
% häufigsten Endzustände bilden.
% DELPHES bietet drei große Jetklassen, welche sich in ihrem Input unterscheiden.
% Zum einen die \textit{Generated Jets}, welche aus langlebigen Teilchen nach
% den Parton-Showern und Hadronisierungsprozessen entstehen. Hierbei werden
% keine Detektorsimulations- bzw. Rekonstruktionsinformationen beachtet.
% Zum Zweiten die \textit{Calorimeter Jets}, welche Kalorimetereinträge als
% Eingabe erhalten und zum Dritten die \textit{particle flow jets}, welche das
% Resultat von Zusammenschlüssen von particle-flow Spuren sind.
% Abschließend werden Informationen über fehlende Transversalenergie
% $\symup{\vec{E}_{T}^{\text{miss}}}$ zurate gezogen, um Aussagen über Teilchen wie die
% Neutrinos zu treffen.
% Dabei ist die fehlende Transversalenergie als
% \begin{equation}
%   \symup{\vec{E}_{T}^{\text{miss}}} = -\sum_{i} \vec{p}_{T}(i)
% \end{equation}
% definiert.


\section{Verzweigungsverh\"altnisse des Λ-Baryons und des $K_{S}$}

% Das Verzweigungsverh\"altnis des dominanten Zerfalls des Kaons
% wird zuerst bestimmt, um einen Konsistenztest mit den Ergebnissen des
% PDG(particle data group) anzufertigen.
% Im Anschluss wird auch das Verzweigungsverhältnis des dominanten
% Zerfalls des $\symup{\Lambda}$-Baryons bestimmt.
% Damit kann die Validit\"at des Datensatzes \"uberpr\"uft werden.
%
% Hierzu werden die Truth-Informationen der Teilchen verwendet.
% Beginnend mit den $K_{S}$ wird ein
% TLV\footnote{TLorentzVector: Memberfunktion aus ROOT \cite{rootyboy}}
% angelegt welcher den Transversalimpuls, das $\eta$, das $\phi$ und eine
% Massenhypothese speichert. Zwischen diesem Kandidaten und dem Jet mit
% dem h\"ochsten $p_{T}$ wird ein Abstand $\symup{\Delta R}$ von 0.5 definiert,
% der nicht \"uberschritten werden darf.
% Darauffolgend wird \"uberpr\"uft, ob beide Tochterteilchen
% entgegengesetzt geladene Pionen sind. Nur diese werden gez\"ahlt.
% Am Ende wird der Wert der $\pi^{+} \pi^{-}$ Endzust\"ande gez\"ahlt
% und durch die Gesamtanzahl geteilt, um das Verzweigungsverhältnis
% zu erhalten.
%
% \"Ahnlich wird f\"ur die $\symup{\Lambda}$-Baryonen verfahren. Es wird
% wieder gez\"ahlt wie viele Tochterteilchenpaare aus
% $p \pi^{-}$ bzw. $\pi^{-} p$ bestehen und wie viele Paare
% insgesamt vorhanden sind.
% Der Bruchteil
% \begin{equation}
%   \text{BR} = \frac{\text{Counts}(p \pi^{-} + \pi^{-} p)}{\text{Counts(all)}}
% \end{equation}
% ist das Verzweigungsverhältnis mit der h\"ochsten Wahrscheinlichkeit
% f\"ur den $\symup{\Lambda}$-Zerfall.
% Dies wird f\"ur jedes der drei generierten Samples
% $\symup{pp}\rightarrow\symup{s\bar{s}}$,
% $\symup{pp}\rightarrow\symup{d\bar{d}}$
% und $\symup{pp}\rightarrow\symup{u\bar{u}}$ durchgef\"uhrt.
%
% Die dazugeh\"origen Messwerte befinden sich in Tabelle \ref{tab:truthBR}.
% \begin{table}
% \centering
% \caption{Zerfallsbreiten f\"ur das $K_S$ und das  $\symup{\Lambda}_0$-Baryon \cite{kaonPDG}, \cite{lambdaPDG}.}
% \label{tab:truthBR}
% \begin{tabular}{c|c c|c c}
% \toprule
% $\text{Sample}$
% & \multicolumn{2}{c}{$\Lambda \rightarrow p\pi^{-}$}
% & \multicolumn{2}{c}{$K_S \rightarrow \pi^{+}\pi^{-}$} \\
% $\text{Prozess}$ & $\text{BR}$ & $\text{BR(PDG)}$
% & $\text{BR}$ & $\text{BR(PDG)}$ \\
% \midrule
% $\text{pp} \rightarrow s\bar{s}$ & $\SI{65.26}{\percent}$ & $\SI{63.90(05)}{\percent}$ & $\SI{70.95}{\percent}$ & $\SI{69.20(05)}{\percent}$ \\
% $\text{pp} \rightarrow d\bar{d}$ & $\SI{64.90}{\percent}$ & $\SI{63.90(05)}{\percent}$ & $\SI{71.11}{\percent}$ & $\SI{69.20(05)}{\percent}$ \\
% $\text{pp} \rightarrow u\bar{u}$ & $\SI{65.04}{\percent}$ & $\SI{63.90(05)}{\percent}$ & $\SI{70.96}{\percent}$ & $\SI{69.20(05)}{\percent}$ \\
% \bottomrule
% \end{tabular}
% \end{table}
%
% Die berechneten Verzweigungsverhältnisse des $\symup{\Lambda}$-Zerfalls
% weichen alle jeweils um $\left(\num{1} - \num{2}\right)\si{\percent}$
% von den Werten im PDG ab. Dies kann an der relativ kleinen Statistik
% liegen. Es wurden in jedem Sample nur eine Millionen Ereignisse
% generiert. Demnach ist diese Abweichung plausibel, sodass
% eine gute Konsistenz mit dem PDG vorliegt.
% Die berechneten Verzeigungsverh\"altnisse des dominanten $\symup{K}_{S}$-Zerfalls weichen auch um wenige Prozent von den PDG-Werten ab.
% Dies liegt wahrscheinlich ebenfalls an der kleinen Statistik und ist dadurch erkl\"arbar.
% Zusammenfassend bedeutet das, dass die Samples valide sind und eine weitere Analyse keine grundlegend falschen Ergebnisse liefern sollte.

\newpage

\section{Ereignisselektion und Objektselektion}
\label{sec:ereignissel}

% Einleitend in die Rekonstruktion des $\symup{\Lambda}$-Baryons ist der dominante
% Zerfall in Abbildung \ref{fig:ldecay} dargestellt.
%
% \begin{figure}
\centering
\begin{tikzpicture}
\begin{feynman}
\vertex (u) {\(u\)};
\vertex[right=5cm of u] (uu) {\(u\)};
\vertex[below=2em of u] (d) {\(d\)};
\vertex[right=5cm of d] (dd) {\(d\)};
\vertex[below=2em of d] (s) {\(s\)};
\vertex[right=5cm of s] (uuu) {\(u\)};
\vertex[below right=1cm and 2.5cm of s] (v1);
\vertex[below right=1cm and 2cm of v1] (v2);
\vertex[above right=0.5cm and 1cm of v2] (ubar) {$\bar{u}$};
\vertex[below right=0.5cm and 1cm of v2] (dnorm) {$\symup{d}$};
\diagram* { {[edges=fermion]
(u) -- (uu),
(d) -- (dd),
(s) -- (v1) -- (uuu),
(ubar) -- (v2) -- (dnorm)},
(v1) -- [boson, edge label=\(W^{-}\)] (v2)
};
\draw [decoration={brace}, decorate] (s.south west) -- (u.north west) node [pos=0.5, left] {\(\Lambda\)};
\draw [decoration={brace}, decorate] (uu.north east) --  (uuu.south east) node [pos=0.5, right] {\(p\)};
\draw [decoration={brace}, decorate] (ubar.north east) --  (dnorm.south east) node [pos=0.5, right] {\(\pi^{-}\)};
\end{feynman}
\end{tikzpicture}
\caption{FEYNMAN Diagramm des dominanten $\symup{\Lambda}$-Baryon Zerfalls.}
\label{fig:ldecay}
\end{figure}

%
% Mein Auswertungsprogramm zur $\symup{\Lambda}$-Rekonstruktion wurde mit dem Hochenergieframework ROOT f\"ur C++ geschrieben, demnach werden viele Methoden aus diesem verwendet.
%
% % charge criteria of track pairs
% Das $\symup{\Lambda}$-Baryon ist ein neutral geladenes Baryon, welches in zwei Tochterteilchen zerf\"allt.
% Demnach m\"ussen diese entgegengesetzt geladen oder beide neutral geladen sein.
%
% Da der Zerfall $\symup{\Lambda} \rightarrow \symup{n}\pi^{0}$ aufgrund von neutral geladenen Spuren des Neutrons und des $\pi^{0}$ nicht von DELPHES simuliert werden kann ist der einzige zu betrachtene Zerfall
% $\symup{\Lambda} \rightarrow \symup{p}\pi^{-}$.
% Demnach werden weiter nur Spurpaare verwendet, welche eine
% multiplizierte Ladung von -1 haben, also unterschiedlich geladen sind.
%
% % reference lambda
% Nun wurde ein \textit{TLV} f\"ur ein $\symup{\Lambda}$-Baryon erstellt. Dieses wird als
% Referenz f\"ur jegliche Vergleiche und Kriterien verwendet.
% Dieser \textit{TLV} enth\"alt Informationen \"uber den Transversalimpuls $\symup{p}_{\symup{T}}$,
% die Pseudorapidit\"at $\eta$, den Azimutalwinkel $\phi$ und eine Massenhypothese.
%
% Im Anschluss werden Kandidaten f\"ur $\symup{\Lambda}$-Baryonen erzeugt.
% Diese bestehen aus zwei Spuren korrespondierend f\"ur zwei Tochterteilchen.
% Erzeugt werden diese wieder mit Hilfe eines \textit{TLorentzVector}
% f\"ur jeweils den ersten und zweiten Eintrag eines Spurpaares.
% In diesen stecken die Hypothesen der Protonenmasse f\"ur Spur 1 und die
% der Pionmasse f\"ur Spur 2.
% Das liegt daran, dass das Proton in den meisten F\"allen mit
% dem h\"oheren Transversalimpuls eingeht \cite{stquelle}.
% Der Kandidat besteht nun aus der Summe der beiden Spuren.
% Dieser weist nun vier Parameter auf. $\symup{p}_{\symup{T}}$, $\eta$, $\phi$ und
% einer Masse, welche aus dem Spurpaar durch Addition beider \textit{TLV} gebildet wird.
% Es werden au\ss erdem nur Kandidaten ausgew\"ahlt, welche einen Abstand
% $\symup{\Delta R}$ zur Jetachse haben, mit $\symup{\Delta R} < \num{0.5}$.
% Dabei ist $\symup{\Delta R} =
% \sqrt{\Delta\phi^{2} + \Delta\eta^{2}}$, $\eta$ ist die
% Pseudorapidit\"at und $\phi$ ist der Azimutalwinkel.
% Der Grund daf\"ur ist ein "geometrisches Matching" des Kandidaten mit
% dem s- oder $\bar{s}$-Jet, weshalb $|\symup{\Delta R}| < \num{0.5}$
% sein muss.
% Zus\"atzlich soll der Transversalimpuls der Kandidaten  gr\"o\ss er als
% $\SI{1}{\giga\electronvolt}$ sein, da daf\"ur die Effizienz des Trackers
% $\SI{95}{\percent}$ ist \cite{jerdmannTagger}.
% Au\ss erdem muss der Transversalimpuls der Jets gr\"o\ss er als
% $\SI{20}{\giga\electronvolt}$ sein, bei einem maximalen
% $\eta(\text{Jet}) = \num{2.5}$.
% Danach werden die PIDs(particle Identification) der beiden Spuren der Kandidaten betrachtet.
% Damit keine $\pi^{+}\pi^{-}$ Endzustände weiterverwendet werden, werden alle
% Spurpaare mit $\text{PDGID}\footnote{particle data group Identification: Jedes
% Teilchen tr\"agt eine individuelle Nummer}(Spur 1) = \pm \num{211}$ und
% $\text{PDGID}(Spur 2) = \mp \num{211}$ verworfen.
% Um zu garantieren, dass keine $\symup{\Lambda}$-Antibaryonen im reduzierten
% Sample verbleiben, werden au\ss erdem die Spurpaare entfernt die
% $\text{PDGID}(\text{Spur} 1) = \text{PDGID}(\text{Antiproton})$ und $\text{PDGID}(\text{Spur} 2) = \text{PDGID}(\pi^{+})$ tragen.

\newpage

\section{Auswahl von Diskriminanten}
\label{sec:disk}
%$\symbf{hier\, muss\, einiges\, nochmal\, neu\, geschrieben\, werden...}$\\

% Aus den in Kapitel \ref{sec:ereignissel} genannten Objekten werden
% die ausgewählt, die eine diskriminierende Wirkung zwischen
% den Datensätzen, zeigen könnten.
%
% Vorüberlegungen ergeben, dass zum ersten $X_{\Lambda}$, also
% $\frac{p_{T}(\Lambda)}{p_{T}(\text{Jet})}$ als Diskriminator
% fungieren könnte.
% Bei der Betrachtung der FEYNMAN-Graphen in den Abbildungen
% \ref{fig:feyn1}, \ref{fig:feyn2} und \ref{fig:feyn3} gibt es einen entscheidenden Unterschied.
% In dem Prozess $pp \rightarrow s\bar{s}$ ist es m\"oglich,
% dass die Strange-Quarks nach der Hadronisierung und Parton-Showern
% Bestandteil eines $\symup{\Lambda}$-Baryons sind.
% Das bedeutet, dass sie ihren vollen Transversalimpuls in das
% $\symup{\Lambda}$ tragen.
% Wohingegen $\symup{\Lambda}$, die nach der Hadronisierung aus dem
% Prozess $pp \rightarrow d\bar{d}(u\bar{u})$ enstehen, ein
% Strange-Quark ben\"otigen.
% Dieses kann nur w\"ahrend der Hadronisierung entstehen.
% Bis sich ein $\symup{\Lambda}$-Baryon gebildet hat, kann das Strange-Quark aber
% noch Teile seines Transversalimpulses abgeben, wodurch im Mittel
% der Gesamttransversalimpuls niedriger sein m\"usste als der von den $\symup{\Lambda}$-Baryonen die durch den Prozess $pp \rightarrow s\bar{s}$ initiiert wurden.
%
% Zum Zweiten wurde $\symup{\Delta R}$, welches der Abstand
% zweier Spuren oder eines gesuchten Teilchens und eines Jets in
% der $\eta$ - $\phi$ - Ebene beschreibt, gew\"ahlt.
% Wenn das in Abbildung \ref{fig:feyn1} erzeugte Strange-Quark f\"ur ein
% $\symup{\Lambda}$-Baryon verwendet wird, liegt der Jet weitestgehend in Richtung der
% Hadronisierung, das bedeutet, dass $\symup{\Delta R}$ in den meisten F\"allen kleiner ist, als wenn ein Strange-Quark aus der Hadronisierung Bestandteil eines $\symup{\Lambda}$-Baryons sein w\"urde.
% Wenn die Quarks, welche gem\"a\ss\, Abbildung \ref{fig:feyn2} und \ref{fig:feyn3}
% erzeugt werden, Bestandteile eines $\symup{\Lambda}$-Baryons werden, werden die Abst\"ande zwischen dem gesuchten Teilchen und dem dazugeh\"origen Jet gr\"o\ss er sein. Somit ergibt sich vermutlich auch eine diskriminierende Wirkung. Die \"Uberlegung, die Masse des $\symup{\Lambda}$ zu verwenden, l\"asst sich leicht als Sackgasse
% identifizieren, da die Verteilungsfunktion der Masse f\"ur alle drei Datens\"atze fast exakt \"ubereinander liegen sollten, da die Masse nicht vom Erzeugungszeitpunkt abh\"angen darf. Demnach ist hier keine diskriminierende Wirkung zu erwarten.
%
% In den Abbildungen \ref{fig:deltaR}, \ref{fig:xlambda} und
% \ref{fig:massL} sind die Plots der jeweiligen Diskriminante
% gegen die H\"aufigkeit aufgetragen. In Abbildung
% \ref{fig:deltaR} ist zu erkennen, dass $\symup{\Lambda}$-Baryonen mit Strange-Quark aus dem Prozess
% $pp \rightarrow s\bar{s}$ weniger weit von der Jetachse abweichen als $\symup{\Lambda}$-Baryonen die ein Strange-Quarks nach der Hadronisierung verwenden.
% Dies bekr\"aftigt auch die Hypothese
% aus Kapitel \ref{sec:disk}. In Abbildung \ref{fig:xlambda} ist
% die Verteilung von $X_{\Lambda}$ gegen\"uber
% der H\"aufigkeit dargestellt. Es ist klar
% zu erkennen, dass $\symup{\Lambda}$-Baryonen, welche aus $pp \rightarrow s\bar{s}$ Prozessen initiiert werden
% h\"aufiger h\"ohere Transversalimpulse haben
% (schwarze Kurve), als solche, die auf die anderen beiden Arten entstehen.
%
% \begin{figure}
%   \centering
%   \includegraphics[width=0.8\textwidth]{../../../../../../bscnilsbreer_new/BA_NilsBreer_2019/output/DeltaR_A3.pdf}
%   \caption{Plot f\"ur den Abstand $\symup{\Delta R}$ zwischen dem Jet in f\"uhrender $\symup{p}_{\symup{T}}$ Ordnung und dem Rekonstruierten $\symup{\Lambda}$-Baryon.}
%   \label{fig:deltaR}
% \end{figure}
%
% \begin{figure}
%   \centering
%   \includegraphics[width=0.8\textwidth]{../../../../../../bscnilsbreer_new/BA_NilsBreer_2019/output/XLambda.pdf}
%   \caption{Plot f\"ur das Transversalimpulsverh\"altnis des $\symup{\Lambda}$-Baryons zum Jet-Transversalimpuls.}
%   \label{fig:xlambda}
% \end{figure}
%
% \begin{figure}
%   \centering
%   \includegraphics[width=0.8\textwidth]{../../../../../../bscnilsbreer_new/BA_NilsBreer_2019/output/LInvMass_true_p_pi.pdf}
%   \caption{Invariante $\symup{\Lambda}$-Baryon-Masse $\sqrt{\symup{s}}$ in $\si{\mega\electronvolt}$.}
%   \label{fig:massL}
% \end{figure}
%
% \newpage

\section{Bestimmung der G\"ute der Diskriminanten}

% Da ein Algorithmus zur Rekonstruktion von $K_{S}$ bereits zur Verf\"ugung
% stand, wurden f\"ur diese ROC-Kurven im Hinblick auf
% die selben Diskriminanten wie f\"ur die $\symup{\Lambda}$-Baryonen erstellt.
% Das Ziel ist es, diese Ergebnisse als Vergleich bzw. f\"ur einen
% Konsistenztest zu verwenden.
%
% Die erstellten Kurven befinden sich in den Abbildungen
% \ref{fig:kaon_roc_sd} und \ref{fig:kaon_roc_su}.
% Aufgetragen auf der y-Achse ist
% die Sensitivit\"at des Strange-Quark Monte-Carlo-Samples, welche
% der Bruchteil der "true positives" \"uber der Gesamtzahl ist.
% Sie ist ein Ma\ss\, f\"ur die Signaleffizienz (SE). Auf der x-Achse ist
% 1-Spezifizit\"at des Down-Quark Monte-Carlo-Samples aufgetragen. Sie entspricht der Hintergrundeffizienz(BE).
% Die rote Kurve stellt die Linie dar, bei welcher der Algorithmus Zufallswerte liefern w\"urde. Liegt die Kurve dar\"uber, arbeitet der Algorithmus gut. Liegt die Kurve darunter arbeitet er dementsprechend schlechter.
%
% In beiden Abbildungen ist zu erkennen, dass die Variablen $X_{K}$ und
% $\symup{\Delta R}$, wie bei den $\symup{\Lambda}$-Baryonen, ziemlich gut
% diskriminieren und gut f\"ur die Rekonstruktion geeignet sind.
% Es ist auch zu sehen, dass mithilfe der Masse keine Aussage bez\"uglich
% der Signaleffizienz getroffen werden kann.
%
% \begin{figure}
%   \centering
%   \includegraphics[width=0.8\textwidth]{../../../../../../bscnilsbreer_new/BA_NilsBreer_2019/py_output/kaon_ROC_s_d.pdf}
%   \caption{ROC Kurve f\"ur den Vergleich zwischen $\text{s}\bar{\text{s}}$ und $\text{d}\bar{\text{d}}$ Sample f\"ur Kaonen.}
%   \label{fig:kaon_roc_sd}
% \end{figure}
%
% \begin{figure}
%   \centering
%   \includegraphics[width=0.8\textwidth]{../../../../../../bscnilsbreer_new/BA_NilsBreer_2019/py_output/kaon_ROC_s_u.pdf}
%   \caption{ROC Kurve f\"ur den Vergleich zwischen $\text{s}\bar{\text{s}}$ und $\text{u}\bar{\text{u}}$ Sample f\"ur Kaonen.}
%   \label{fig:kaon_roc_su}
% \end{figure}
%
% Um nun etwas \"uber die Aussagekr\"aftigkeit der Diskriminanten
% bez\"uglich des $\symup{\Lambda}$-Baryons zu sagen,
% wurden auch ROC-Kurven, zusehen in
% den Abbildungen \ref{fig:roc_sd} und \ref{fig:roc_su}
% %\cite{roccy1}, \cite{roccy2}
% angefertig, welche die Ergebnisse veranschaulichen und
% G\"ute der Diskriminanten validieren sollen. Dazu wurden die
% Ergebnisse resultierend aus dem Datensatz mit $s\bar{s}$ im
% Endzustand mit jeweils einem der beiden anderen verglichen.
%
% Dargestellt sind die Diskriminanten $X_{\Lambda}$,
% $\symup{\Delta R}$ und $M_{\Lambda}$.
% An der schwarzen Kurve ist zu erkennen, dass sie relativ weit \"uber der
% roten Kurve liegt, $X_{\Lambda}$ also als diskriminierende Variable
% zu gebrauchen ist.
% Es ist auch zu erkennen, dass bei kleiner werdender SE die BE auch kleiner wird, doch in einem st\"arkeren Ma\ss.
%
% Die blaue Kurve von $\symup{\Delta R}$ liegt auch noch \"uber
% der roten Kurve, aber nicht mehr so weit wie die schwarze. $\symup{\Delta R}$
% ist also noch als Diskriminator zu verwenden.
% Die gr\"une Kurve liegt sehr genau auf
% der roten Kurve, weshalb diese Variable nicht als Diskriminante
% zu verwenden ist.
%
% Wie zu erwarten arbeiten $\symup{\Delta R}$ und $X_{\Lambda}$
% gut als Diskriminanten.
%
% \newpage
%
% \begin{figure}
% \centering
% \begin{subfigure}{\textwidth}
%   \centering
%   \includegraphics[width=0.8\textwidth]{../../../../../../bscnilsbreer_new/BA_NilsBreer_2019/py_output/ROC_s_d.pdf}
%   \caption{ROC-Kurve f\"ur $\symup{\Lambda}$-Baryonen zwischen Samples mit $s\bar{s}$ Endzustand und $d\bar{d}$ Endzustand.}
%   \label{fig:roc_sd}
% \end{subfigure}
% \begin{subfigure}{\textwidth}
%   \centering
%   \includegraphics[width=0.8\textwidth]{../../../../../../bscnilsbreer_new/BA_NilsBreer_2019/py_output/ROC_s_u.pdf}
%   \caption{ROC-Kurve f\"ur $\symup{\Lambda}$-Baryonen zwischen Samples mit $s\bar{s}$ Endzustand und $u\bar{u}$ Endzustand.}
%   \label{fig:roc_su}
% \end{subfigure}
% \caption{ROC-Kurven f\"ur $\symup{\Lambda}$-Baryonen.}
% \label{fig:ROCCY}
% \end{figure}
