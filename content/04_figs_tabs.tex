\chapter{Abbildungen und Tabellen}
\begin{figure}
    \centering
    \includegraphics[scale=1]{./Plots/Histogramm.pdf}
    \caption{Ein Histogramm mit Fehlerbalken für zwei Datensätze, Schriftgröße und -art entsprechen der des Dokuments.}
    \label{fig:bsp}
\end{figure}
\begin{table}
    \centering
    \caption{Beispieltabelle mit willkürlichen Werten, für die Zahlenwerte wurde die S-Option aus \texttt{siunitx} verwendet.}
    \label{tab:bsp}
    \begin{tabular}{S[table-format=4.2] S[table-format=3.2]}
        \toprule
        {$p \mathrel{/} \si{\pascal}$}  & {$T \mathrel{/} \si{\kelvin}$} \\
        \midrule
        1024,23 & 273,15 \\
        1025,31 & 274,5 \\
        1026,27 & 276,2 \\
        \bottomrule
    \end{tabular}
\end{table}
