\chapter{The LHCb Experiment}
\label{sec:theorie}

\section{the Scintillating Fibre Tracker}
\begin{enumerate}
  \item layout
  \item how does it work?
  \item what else?
\end{enumerate}

\chapter{Alignment}
martinelli pdf! use some of that information
-> alignment is a minimizing problem (chi2) thats why i looked at chi2 plots

-> global translation and sheering motion don't change chi2 values because residuals are unchanged.

-> weak modes: presence of weak modes affect the convergence (poor, takes many iterations), bias in track parameters.

-> most visible weak modes is the "curvature bias" (sophie has mentioned it sometime. must be on one of my sheets)
also look at twiki!
\section{what is alignment used for?}
short answers:
\begin{enumerate}
  \item yielding the best possible reconstruction efficiency
  \item getting a good feeling for particle masses
  \item momentum resolution of the detector improved
\end{enumerate}

for these 3 bullet points i need a subsection explaining it!

\section{when does alignment happen?}
at which point during a run will alignment come into play?

\section{Alignment Methods}

\subsection{using tracks fitted with kalman}
talk pdf. quelle herausfinden!
\subsection{'global' alignment with collision data}
wouter pdf. quelle herausfinden!

\section{Alignment goals}
source for now: DPG2021 pdf exact source will be included!
\begin{enumerate}
  \item find the best possible configuration of alignables, degrees of freedom, constraints
  \item check for weak modes (how? chi2, small eigenvalues)
  \item null tests as good as possible
  \item misalignment tests to check alignment
\end{enumerate}

all of tye above is just theory. Now, the story i want to tell starts.
