\chapter{Theorie}
\label{sec:theorie}
\section{Einleitung in das Standardmodell mit Fokus auf die Quarks}

Das Standardmodell der Teilchenphysik beschreibt die bekannten
Elementarteilchen und ihre Wechselwirkungen.
Es besteht aus 12 Materieteilchen, den Fermionen und f\"unf
Wechselwirkungsteilchen, welche Vektorbosonen genannt werden.
Die Fermionen lassen sich in zwei Kategorien zusammenfassen,
zu sechs Quarks und sechs Leptonen.
Sowohl die Quarks als auch die Leptonen werden in drei Generationen
aufgeteilt.
Jedes Materieteilchen besitzt au\ss erdem ein Antiteilchen, welches eine
entgegengesetzte Ladung besitzt.

\subsection{Leptonen}
Spin-$\frac{1}{2}$-Fermionen nennt man Leptonen, wenn sie nicht \"uber die
starke Wechselwirkung wechselwirken.
Die drei Generationen $\symup{e}$, $\symup{\mu}$ und $\symup{\tau}$
nennt man auch \textit{Flavor}.
In jeder dieser Familien gibt es ein geladenes Lepton, welches in beiden
H\"andigkeiten vorkommmt und ein dazugeh\"origes linksh\"andiges Neutrino.
Ein Teilchen hei\ss t linksh\"andig, wenn seine Spinrichtung entgegengesetzt
der Flugrichtung zeigt.
Rechtsh\"andige Teilchen haben einen Spinrichtung die mit der
Flugrichtung zeigt.
Pro Familie bildet sich so ein linksh\"andiges Isospindublett und
rechtsh\"andiges Singulett.
Neutrinos wechselwirken, anders als die geladenen Leptonen,
nur \"uber die schwache Wechselwirkung.

\subsection{Quarks}
Die Quarks sind Spin-$\frac{1}{2}$-Teilchen und
tragen eine elektrische Ladung.
In jeder der drei Generationen gibt es ein Isospindublett.
Die Quarks sind nach aufsteigender Masse geordnet.
In der ersten Generation stehen die beiden leichtesten Quarks,
up-und down-Quark in der zweiten Generation das charme-und Strange-Quark
und in der dritten Generation das top-und bottom-Quark Dublett.

Quarks tragen eine Farbladung, rot, gr\"un oder blau, welche ein
k\"unstlich eingef\"uhrter Freiheitsgrad ist, um die
Unterscheidbarkeit zu garantieren.
Es existieren neun Quarkfarbzust\"ande, welche aus einem Oktett und
einem Singulett zusammengesetzt sind.
Das farblose oder wei\ss e Singulett geh\"ort nicht zu den
Generatoren der Farbgruppe,
da es keine Farbe in der Wechselwirkung \"ubertragen kann.
Das Oktett hingegen bildet die Generatoren der SU(3)
(spezielle unit\"are Gruppe der Dimension 3) welche aus den
acht Quarkfarbzust\"anden generiert wird.

Da es keine freien Quarks gibt, finden sie sich zu
Bindungszust\"anden zusammen, den sogenannten Hadronen.
Zum einen gibt es die Mesonen, welche aus einem Quark und
einem Antiquark bestehen.
\begin{equation}
	|\symup{M}\!> = | q\, \bar{q}'\!>
\end{equation}
Diese k\"onnen aus der gleichen Familie (d.h [u,d], [c,s], [t,b])
stammen,
oder aus unterschiedlichen Familien kommen.
Sie haben die Baryonenzahl 0 und m\"ussen immer farbneutral sein.
Demnach tragen Quarks die Baryonenzahl $\frac{1}{3}$.
Das bedeutet die beiden Quarks eines Mesons m\"ussen Farbe und die
korrespondierende Antifarbe tragen.
Das hier wichtige Meson ist das $K_S$. Es hat den Quarkinhalt
\begin{equation}
\symup{K}_S = |u \bar{s}\!>\,.
\end{equation}
Die zweite Art sind die Bayonen.
Der Inhalt besteht entweder aus drei Quarks oder drei Antiquarks.
Es kann jedoch nicht sein, dass ein Quark und zwei
Antiquarks und andersherum auftreten, da Baryonen
die Baryonenzahl $B = 1$ haben m\"ussen.
In einem Baryon m\"ussen alle Farben enthalten sein, da sie wie die
Mesonen in Summe wei\ss\, sein m\"ussen.
Sie k\"onnen allgemein als
\begin{align}
	|\symup{B}\!> &= |q q' q''\!> \\
	|\symup{B}\!> &= |\bar{q} \bar{q}' \bar{q}''\!> \,
\end{align}
beschrieben werden.

Das $\Lambda_{0}$-Baryon ist hier das wichtigste Baryon. Es
hat den Quarkinhalt
\begin{equation}
\symup{\Lambda}_0 = |u d s\!>\,.
\end{equation}

\section{Die Wechselwirkungen}
Die Wechselwirkungen werden aus den oben genannten Vektorbosonen gewonnen.
Es existieren drei relevante Wechselwirkungen. Die elektromagnetische
Wechselwirkung, die schwache Wechselwirkung und die starke
Wechselwirkung. Die Gravitation wird hier nicht betrachtet, da sie keinen
nennenswerten Beitrag liefert.

Unter der em(elektromagnetischen)-Wechselwirkung versteht
man den Austausch eines Photons zwischen den Teilchen.
Die St\"arke einer jenen Wechselwirkung wird durch eine
Kopplungskonstante beschrieben. In der em-Wechselwirkung ist
dies die Feinstrukturkonstante \cite{alphas}.
Die Reichweite der em-Wechselwirkung ist prinzipiell unendlich,
f\"allt aber mit zunehmendem Abstand zwischen den
wechselwirkenden Teilchen ab.

Die em-Wechselwirkung wird durch die Quantenelektrodynamik
beschrieben. Dort werden die Potenziale als Operatoren aufgefasst.
Mit den Erzeugern und Vernichtern k\"onnen so die Photonen, also die
Wechselwirkungsteilchen erzeugt, und vernichtet werden.

Die Austauschteilchen der schwachen Wechselwirkung sind zum einen die $\symup{W}^{\pm}$-Bosonen
und zum anderen das Z-Boson.
Man spricht in der schwachen Wechselwirkung von Str\"omen.
Wenn in einem Prozess die Ladung zum Beispiel eines Teilchens
durch ein W-Boson ver\"andert wird, wird dies geladener Strom genannt.
Wohingegen die Austauschreaktion eines Z-Bosons bei zum
Beispiel Prozessen wie
$\symup{e} \nu_{\mu} \rightarrow \symup{e} \nu_{\mu}$ neutraler Strom
genannt wird.

Analog zur elektromagnetischen Wechselwirkung werden die Potenziale wieder
als Operatoren verstanden, doch hier gibt es keine Propagatoren.
Propagatoren werden in FEYNMAN-Diagrammen der QED verwendet, um die
Wechselwirkungsteilchen darzustellen.
Es wird hier stattdessen eine sogenannte V-A Struktur verwendet.
\begin{align}
	V &= \text{Vektoren}\\
	A &= \text{Axialvektoren}
\end{align}
Diese Struktur wird ben\"otigt um die rechtsh\"andigen Teilchen und
linksh\"andigen Antiteilchen nicht zu beachten, da diese zur
Parit\"atsverletzung  f\"uhren bzw. Ladungserhaltung verletzen.
Man passt somit die Lorentzfaktoren auf folgende Weise an
\begin{equation}
	\gamma^{\mu} \rightarrow \gamma^{\mu}\left(1 - \gamma^{5}\right)\,.
\end{equation}

Bei der starken Wechselwirkung sind die Austauschteilchen
die acht verschiedenen Gluonen.
Die starke Wechselwirkung wird durch die Quantenchromodynamik
(QCD) beschrieben. Bei der Wechselwirkung wird demnach Farbe \"ubertragen.
Gluonen haben keine Masse und sind Spin-1-Teilchen.
Gluonen tragen jeweils eine Farbe und eine Antifarbe, welche nicht
zwingend identisch sein m\"ussen. Gluonen k\"onnen, anders als
Photonen, auch an sich selbst koppeln. Au\ss erdem ist die Kopplungskonstante
$\alpha_{s}$ circa $\num{0.1}$. Die Wechselwirkung mit Quarks wird
mit einem Potential \cite{cornellPot} der Form
\begin{align}
\symup{V}_{q\bar{q}} &= -\frac{4 \alpha_{s}}{3 r} + \sigma\cdot r
\intertext{mit}
\sigma &= \SI{1}{\giga\electronvolt\per\femto\metre}
\end{align}
beschrieben. Quarks tendieren demnach sehr stark dazu sich anzuziehen.
Wenn nun Quark und Antiquark voneinander entfernt werden, muss viel
Energie aufgewendet werden. Diese kann so gro\ss\, werden, dass neue
Teilchen entstehen k\"onnen.

% maybe einfach auf englisch?
