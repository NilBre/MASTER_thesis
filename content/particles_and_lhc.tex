\chapter{Particles and The Large Hadron Collider}
\label{sec:particleslhc}

% Here needs to be an introduction to the matter. why we do what we do.
% why do we need to know about the SM particles? -> check \\
% Why do we want to build an LHC or an LHCb upgrade at all? -> check \\
% What can it help us to learn? -> need to think about it \\
% remember!! a big introduction is also needed in the beginning but here needs to be a small one too.\\

Whenever a new detector is build the position where it is physically mounted is
roughly where it should be. In order to check this, survey measurements are
performed to check the position with a precision of $\SI{100}{\micro\metre}$.
To achieve an even bigger precision, \textit{software alignments} are performed.

The reason why alignment is of great importance is that a misaligned detector
yields large momentum resolutions, low reconstruction efficiencies and wrong
mass estimations.
The most prominent area of misalignment is asymmetries for a spectrometer.
In the past, alignment solved problems for example a Muon asymmetry in the
L0Muon trigger in 2011 and a misalignment in IT boxes which resulted in trigger inefficency regarding $J/\Psi$ in 2012.

With good alignment, studies on all standard modell particles and hadron states
will improve. Yielding better efficiencies on SM particle measurements will
result in deeper insights regarding physics beyond the stand modell from high-precision measurements of CP-violating observables as an example.

% maybe change the order around?
On a grand scheme upgrading the LHCb and therefore the LHC will bring deeper
insights "for" (wrong word) standard model processes. To understand the universe even better and eventually yield information about the unsolved question of "dark matter" and
"dark energy", what is believed to be the bulk of the universes content.

\section{The Standard Modell}
\label{sec:sm}

\begin{figure}
  \centering
  \includegraphics[width=0.75\textwidth]{plots/SM_2018.png}
  \caption{The standard model of particle physics.}
  \label{fig:sm2018}
\end{figure}

The standard model of particle physics\ref{fig:sm2018} describes the known elementary particles and their interactions. It consists of 12 matter particles, the fermions
and five interaction particles, which are called vector bosons.

The fermions 12 spin-$\frac{1}{2}$ particles. Six are called leptons and they are sorted into three families, also called flavors (e, $\mu$ and $\tau$) and six are called quarks. Each of those lepton families has a charged lepton\footnote{can have both h\"ndigkeiten} and a left-handed neutrino.
A particle is called left-handed if its spin direction is opposite to the direction of flight. Right-handed particles have a spin direction pointing with the direction of flight.
Neutrinos can only be left-handed since there is no system where the neutrino can be "overtaken" so the momentum switches and therefore the helicity.
A left-handed isospin doublet and a left-handed singlet can be constructed.
The leptons can couple via the weak-interaction and if they are charged, also via the em-interaction. Neutrinos can only couple via the weak interaction.
Each matter particle also has an antiparticle, with an opposite charge.

The quarks are spin-$\frac{1}{2}$-fermions and carry an electric charge as well. In each of the three generations there is one isospin doublet. The quarks are ordered by ascending mass. In the first generation are the two lightest quarks,
up- and down quark, in the second generation the charm- and strange quark and
in the third generation the top- and bottom quark doublet.
Quarks carry a color charge, red, green or blue, which is an artificially introduced degree of freedom to guarantee the distinguishability.

%---------- interactions -----------
The interactions are obtained from the vector bosons mentioned above.
The three potent interactions are the electromagnetic(em) interaction,
the weak interaction and the strong interaction. Gravitation does not make a significant contribution.
The vector boson of the em interaction is the photon which is exchanged between particles.
% here, a feynman diagram would be nice. as well as for every other force
The strength of one of those interactions is
described by a coupling constant. In the em interaction this is the
fine structure constant\cite{alphas}. The range of the em-interaction is in principle
infinite, but decreases with increasing distance between the interacting particles.
The em interaction is described by quantum electrodynamics.
The potentials are described by operators, which create and annihilate the photons.

The exchange particles of the weak interaction are on the one hand the $W^{\pm}$-bosons and on the other hand the Z-boson.
The weak interaction processes are called currents.
Changing the charge during the interaction by a W-boson is called charged current.
The exchange reaction of a Z boson in, for example, processes such as $e_{\nu} \mu \to e_{\nu} \mu$ is called neutral current.
Analogous to the electromagnetic interaction, the potentials are again understood as
operators, but here there are no propagators. Propagators are
used in \textit{FEYNMAN}-diagrams of QED to represent the interaction particles.
A so-called V-A structure is used here instead. Here, V stands for vectorboson and A is the axialvector.
This structure is needed to disregard the right-handed particles and left-handed
antiparticles, since these lead to the charge-parity violation. Thus the Lorentz factors are adjusted in the following way
\begin{equation*}
  \gamma_{\mu} \to \gamma_{\mu}(1 - \gamma_5)
\end{equation*}
Quarks couple via the strong interaction which is described by the quantum chromodynamic (QCD). The Eichgroup of the QCD is $SU\left(N = 3\right)$ where N is the number of introduced colors as a new degree of freedom. The number of generators is therefore $N^2 - 1 = 8$.
The generators are called gluons and they carry color and anticolor, have no mass and carry spin 1.
Gluons can, other than photons, couple to themselves.
Moreover, the coupling constant $\alpha_s \approx 0.1$. The interaction with quarks is described with a potential.
\begin{align}
  \symup{V}_{q\bar{q}} &= -\frac{4 \alpha_{s}}{3 r} + \sigma\cdot r
  \intertext{with}
  \sigma &= \SI{1}{\giga\electronvolt\per\femto\metre}
\end{align}
The common eight gluon-wavefunctions\cite{qcd} are
\begin{align*}
  \psi_1 &= |r\bar{g}> & \psi_2 &= |r\bar{b}> \\
  \psi_3 &= |g\bar{r}> & \psi_4 &= |g\bar{b}> \\
  \psi_5 &= |b\bar{r}> & \psi_6 &= |b\bar{g}> \\
  \psi_7 &= \frac{1}{\sqrt{2}}\left(|r\bar{r}> - |g\bar{g}>\right) & \psi_8 &= \frac{1}{\sqrt{6}}\left(|r\bar{r}> + |g\bar{g}> - 2|b\bar{b}>\right) \\
\end{align*}

The second wavefunction describes a gluon interaction with a blue quark and changing the color to red.

Quarks thus tend to attract each other very strongly. If now
quark and anti quark are moved away from each other, a lot of energy has to be expended. This energy can become so large that new particles can be created.

Due to the Confinement, quarks cannot exist alone. Instead they form bonding states, so called hadrons. On the one hand there are the mesons, which consist of a quark
and an antiquark.

\begin{equation}
	|\symup{M}\!> = | q\, \bar{q}'\!>
\end{equation}

These may be from the same family (i.e. [u,d], [c,s], [t,b]), or from
different families. Mesons have a baryon number of 0. Accordingly, quarks carry the baryon number $\frac{1}{3}$. The quarks constructing a meson therefore carries color and the corresponding anticolor.
The second type are baryons. The content consists of either three quarks or
three antiquarks. However, it cannot be that one quark and two antiquarks
and vice versa occur, because baryons must have the baryon number $\symup{B} = 1$. Because baryons are stable final states as well as the mesons, the sum of their quark colors must be white. Therefore, every (anti)color must occur once in a baryon.

\begin{align}
	|\symup{B}\!> &= |q q' q''\!> \\
	|\symup{\bar{B}}\!> &= |\bar{q} \bar{q}' \bar{q}''\!> \,
\end{align}

\section{particle decays and hadrons}
\label{sec:decays}
% For you, this is because the bending and momentum of particles (and the location where they decay) is important to the way we can align things. Which sorts of particles can produce long tracks? etc.

\section{The LHC and LHCb}
\label{sec:lhcandB}

\subsection{The LHC}
The Large Hadron Collider (LHC)\cite{lhcInfo} is the most powerfull particle-accelerator on planet earth. With a circumference of $26,7\si{\kilo\metre}$ it is also the longest ring accelerator and it lies between $45\si{\metre}$ and $170\si{\metre}$ below the surface near Geneva in Swizerland. The tunnel was constructed for the LEP experiment between 1984 and 1989 and is operated by the European Organization for Nuclear Research (CERN). The LHC can produce centre of mass energies of $\sqrt{s} = \SI{13}{\tera\electronvolt}$ in proton-proton collisions during Run 2. After the upgrade the LHC will collide particles with the centre of mass energy of around $\sqrt{s} = \SI{14}{\tera\electronvolt}$.
An image of the accelerators and the experiments is shown in fig. \ref{fig:CERN}\cite{facilityCERN}.

\begin{figure}
  \centering
  \includegraphics[angle=-90, origin=c, width=0.7\textwidth]{plots/CERN_layout.pdf}
  \caption{an overview of the LHC facilities.}
  \label{fig:CERN}
\end{figure}

By ionizing hydrogen gas, protons are created and accelerated to $\SI{50}{\mega\electronvolt}$ by the linear accelerator (LINAC 2). Afterwards the beam is injected into the Proton Syncrotron and the Super Proton Synchrotron to a maximum of $\SI{450}{\giga\electronvolt}$ before the beam is brought into the LHC.
The beam containts several bunches with around $\num{1.15e11}$ protons per bunch and a bunch spacing of $\SI{25}{\nano\second}$.%., which is a collision rate of $\SI{40}{\mega\hertz}$.
The LHC houses four major experiments. ATLAS and CMS are classified as general purpose detectors with a detection range of close to $4\pi$. The interaction in these detectors is located in the very center so that tracks going in every direction can possibly be found. Searches for the Higgs Boson is just one of many physics aspects these detectors are build for.
The other two Experiments located at the LHC are ALICE and LHCb.
The ALICE experiment mainly studies the quark-gluon plasma during the runs with lead ion collisions instead of protons.
In this thesis the Scintillating Fibre Tracker (SciFi Tracker) located at the LHCb will be focused at and discussed on the following chapters.

\subsection{The LHCb}

\begin{figure}
  \centering
  \includegraphics[width=0.75\textwidth]{plots/LHCb_facility.jpg}
  \caption{a sideview of the LHCb experiment.}
  \label{fig:LHCb}
\end{figure}

For high energies, b- and $\bar{b}$-hadrons are heavily produced in a tight forward
direction\ref{fig:bbforward}\footnote{They are also produced in a tight backward
direction but the experiment is only build for the forward cone.}.
The LHCb experiment\cite{lhcbInfo} is a forward spectrometer covering
$2 \,<\, \eta \,<\, 5$ in the pseudorapidity range. This experiments main
physics goal is beauty quark physics. A sideview of
the LHCb is shown in figure \ref{fig:LHCb}.
The LHCb consists of several smaller detector components namely the Vertex Locator
(VELO) right on the intercation point, two Ring Imaging Cherenkov counter
(RICH 1 and RICH 2), in front of the spectrometers lies the Trigger Tracker and
behind them the SciFi Tracker which is the important part of this thesis. Further
back a Preshower (PS) is mounted followed
by the electromagnetic calorimeter (ECAL) and the hadronic calorimeter (HCAL).
In the very back, several muon chambers are mounted for every track that is yet
to be determined.

% more explanations needed still

\begin{figure}
  \centering
  \includegraphics{plots/bbar_forward.png}
  \caption{$b$ and $\bar{b}$-hadrons produced heavily forward.}
  \label{fig:bbforward}
\end{figure}

In this section, a general overview about the requirements for the SciFi Tracker as well as the layout will be discribed based on the presentation in the \textit{technical design report}\cite{scifiInfo} of the upgrade.

The upstream and downstream trackers provide a good precision estimate of the momentum of charged particles so that mass resolution of decayed particles can be precisely measured.
% this is a sentence i used from the TDR!
For particle identification the reconstructed trajectories of charged particles are used as input for the RICH detectors.
The limiting factor for the momentum resolution is multiple scattering for tracks with a momentum lower than $\num{80}frac{\text{GeV}}{\text{c}}$. For tracks with a higher momentum the detector resolution is the limiting factor.

% why is the scifi there
The SciFi Tracker replaced the inner Tracker (IT) and the outer Tracker (OT)
and is located in the same place as the downstream trackers that were previously installed.

% data and facts
The instantaneous luminosity after the upgrade is expected to be $\SI{1e33}{\per\centi\metre\squared\per\second}$ to $\SI{2e33}{\per\centi\metre\squared\per\second}$. The bunch spacing will be $\SI{25}{\nano\second}$ and the number of proton-proton interactions per bunch crossing will be $\nu = 3.8$ during the ramp-up phase of the LHC and $\nu = 7.6$ during the active phase.

% layout
\subsection{Layout of the SciFi Tracker}

\begin{figure}
  \centering
  \includegraphics{plots/SciFi_Tracker.png}
  \caption{view of the stations and readout order of the modules.}
  \label{fig:scifi}
\end{figure}

The SciFi Tracker consists of three (T-)stations T1, T2 and T3 with each having four layers ($X1, U, V, X2$). The orientation of these planes with respect to the vertical axis are ($\SI{0}{\degree}, \SI{+5}{\degree}, \SI{-5}{\degree}, \SI{0}{\degree}$).
Since the beampipe is not exactly parallel to the ground the vertical axis is
defined as vertical on the z-axis of the beampipe.
The tilted layers are called stereo layers and serve the purpose of 3D hit localization.
The layers are $\SI{20}{\milli\metre}$ apart from each other in $z$-direction within each station.
Each layer has four quarters with each quarter having five\footnote{six for the last (T-)station.} modules. Each module is constructed from four fibre mats.
A sideview of the SciFi Tracker is displayed in figure\ref{fig:scifi}.
The global coordinate system used is of right-handed nature with positive $z$ pointing away from the interaction point following the beam direction. positive $y$ points upwards, towards the surface and positive $x$ and negative $x$ are defined as A-Side and C-Side\cite{scifiInfo}.
For readout purposes the top and bottom half of each element have inverted x- and y-axis.

To ensure an optimal alignment, a well known geometry is key. Therefore, the
fibres must be aligned within $\SIrange{50}{100}{\micro\metre}$ in $x$-direction and must not be more than $\SI{300}{\micro\metre}$ bent in $z$-direction.
% here a picture of the scifi

\subsection{Scintillating Fibres}
The scintillating fibre material is a polymer with an organic fluorescent dye
added to the polystyrene structure to enhance the yield during the scintillation process.
In order to produce and register a photon signal, the ionization energy is deposited
in the fibre core firstly. The amount of energy need for the polymer to reach an
excited state is just a few electronvolts. The added dye has the particular structure
to match the excitation energy. The energy is transferred via the Förster
Transfer(source needs to be added). The dye generates excited energy states when
particles hit the fibre and deposit their energy.
The long fibre mats and the refractive index make sure total reflection on the inside
happens which guides the photons to the SiPMs. On the opposite end of the SiPM a
full reflective mirror is mounted so the photons travelling to the other end do not
get lost but reflected towards the SiPM.

%\section{The LHC data cycle}
%\label{sec:datacycle}
