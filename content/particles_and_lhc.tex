\chapter{Particles and The Large Hadron Collider}
\label{sec:particleslhc}

\section{The Standard Modell}
\label{sec:sm}

The standard model of particle physics describes the known elementary particles and their interactions. It consists of 12 matter particles, the fermions
and five interaction particles, which are called vector bosons. The
fermions can be grouped in two categories, six quarks and
six leptons. The quarks as well as the leptons are divided into three generations.
Each matter particle also has an antiparticle, with an opposite charge.
The interactions are obtained from the vector bosons mentioned above.
The three potent interactions are the electromagnetic(em) interaction,
the weak interaction and the strong interaction. Gravitation does not make a significant contribution.
The vector boson of the em interaction is the photon which is exchanged between particles.
% here, a feynman diagram would be nice. as well as for every other force
The strength of one of those interactions is
described by a coupling constant. In the em interaction this is the
fine structure constant\cite{alphas}. The range of the em-interaction is in principle
infinite, but decreases with increasing distance between the interacting particles.
The em interaction is described by quantum electrodynamics.
The potentials are described by operators, which create and annihilate the photons.

The exchange particles of the weak interaction are on the one hand the $W^{\pm}$-bosons and on the other hand the Z-boson. The weak interaction processes are called currents.
Changing the charge during the interaction by a W-boson is called charged current.
The exchange reaction of a Z boson in, for example, processes such as $e_{\nu} \mu \to e_{\nu} \mu$ is called neutral current.
Analogous to the electromagnetic interaction, the potentials are again understood as
operators, but here there are no propagators. Propagators are
used in \textit{FEYNMAN}-diagrams of QED to represent the interaction particles.
A so-called V-A structure is used here instead. Here, V stands for vectorboson and A is the axialvector.
This structure is needed to disregard the right-handed particles and left-handed
antiparticles, since these lead to the charge-parity violation. Thus one adjusts the Lorentz factors in the following way
\begin{equation*}
  \gamma_{\mu} \to \gamma_{\mu}(1 - \gamma_5)
\end{equation*}
In the strong interaction, the exchange particles are the eight different
gluons. The strong interaction is described by quantum chromodynamics (QCD).
According to this, color is transfered during the interaction. Gluons
have no mass and are spin-1 particles. Gluons carry a color and a different anti-color. Gluons can also couple to themselves. Moreover, the coupling constant $\alpha_s \approx 0.1$. The interaction with quarks is described with a potential.
\begin{align}
  \symup{V}_{q\bar{q}} &= -\frac{4 \alpha_{s}}{3 r} + \sigma\cdot r
  \intertext{mit}
  \sigma &= \SI{1}{\giga\electronvolt\per\femto\metre}
\end{align}
Quarks thus tend to attract each other very strongly. If now
quark and antiquark are moved away from each other, a lot of energy has to be expended. This energy can become so large that new particles can be created.

% now particles
Spin-$\frac{1}{2}$-fermions are called leptons, if they do not interact via the strong interaction. The three generations e, μ and τ are also called flavors.
In each of these families there is a charged lepton, which is present in both (h\"andigkeiten) and a corresponding left-handed neutrino. A particle
is called left-handed if its spin direction is opposite to the direction of flight.
Right-handed particles have a spin direction pointing with the direction of flight. Per
family, a left-handed isospin doublet and a right-handed singlet are formed.
Neutrinos interact, unlike the charged leptons, only via the weak
interaction.
The quarks are spin-$\frac{1}{2}$-particles and carry an electric charge. In each of the
three generations there is one isospin doublet. The quarks are ordered by ascending
mass. In the first generation are the two lightest quarks,
up and down quark in the second generation the charm- and strange quark and
in the third generation the top- and bottom quark doublet.
Quarks carry a color charge, red, green or blue, which is an artificially introduced degree of freedom to guarantee the distinguishability. There are eight
quark color states.
%There is also a nineth state, costructed by the superposition of all three colors resulting in a colorless state. generatoren der SU(3) = 3^2 - 1 = 8
%are composed of an octet and a singlet. The colorless or white singlet does not belong to the generators of the color group.
%color group, because it cannot transfer any color in the interaction.
The octet forms the generators of the SU(3) (special unitary group of dimension
3) which is generated from the eight quark color states.
Since there are no free quarks, they come together to form bonding states, the
so called hadrons. On the one hand there are the mesons, which consist of a quark
and an antiquark.

\begin{equation}
	|\symup{M}\!> = | q\, \bar{q}'\!>
\end{equation}

These may be from the same family (i.e. [u,d], [c,s], [t,b]), or from
different families. Mesons have a baryon number of 0. Accordingly, quarks carry the baryon number $\frac{1}{3}$. The quarks constructing a meson therefore carry color and the corresponding anticolor.
The second type are baryons. The content consists of either three quarks or
three antiquarks. However, it cannot be that one quark and two antiquarks
and vice versa occur, because baryons must have the baryon number $\symup{B} = 1$. Because baryons are stable final states as well as the mesons, the sum of their quark colors must be white. Therfore, every (anti)color must occur once in a baryon.

\begin{align}
	|\symup{B}\!> &= |q q' q''\!> \\
	|\symup{B}\!> &= |\bar{q} \bar{q}' \bar{q}''\!> \,
\end{align}

\section{particle decays and hadrons}
\label{sec:decays}
For you, this is because the bending and momentum of particles (and the location where they decay) is important to the way we can align things. Which sorts of particles can produce long tracks? etc.

\section{The LHC and LHCb}
\label{sec:lhcandB}

\subsection{The LHC}
The Large Hadron Collider (LHC)\cite{lhcInfo} is the most powerfull particle-accelerator on planet earth. With a circumference of $26,7\si{\kilo\metre}$ it is also the longest ring accelerator and it lies between $45\si{\metre}$ and $170\si{\metre}$ below the surface near Geneva in Swizerland. The tunnel was constructed for the LEP experiment between 1984 and 1989 and is operated by the European Organization for Nuclear Research (CERN). The LHC can produce centre of mass energies of $\sqrt{s} = \SI{13}{\tera\electronvolt}$ in proton-proton collisions during Run 2. After the upgrade the LHC will collide particles with the centre of mass energy $\sqrt{s} = \SI{13}{\tera\electronvolt}$.
An image of the accelerators and the experiments is shown in fig. \ref{fig:CERN}\cite{facilityCERN}.

\begin{figure}
  \centering
  \includegraphics[angle=-90, origin=c, width=0.5\textwidth]{plots/CERN_layout.pdf}
  \caption{an overview of the LHC facilities.}
  \label{fig:CERN}
\end{figure}

By ionizing hydrogen gas, protonsare created and accelerated to $\SI{50}{\mega\electronvolt}$ by the linear accelerator (LINAC 2). Afterwards the beam is injected into the Proton Syncrotron and the Super Proton Synchrotron to a maximum of $\SI{450}{\giga\electronvolt}$ before the beam is brought into the LHC.
The beam containts several bunches with around $\num{1.15e11}$ and a bunch spacing of $\SI{25}{\nano\second}$, which is a collision rate of $\SI{40}{\mega\hertz}$.
The LHC houses four major experiments. ATLAS and CMS are classified as general purpose detectors with a detection range of close to $4\pi$. The interaction in these detectors is located in the very center so that tracks going in every direction can possibly be found. Searches for the Higgs Boson is just one of many physics aspects these detectors are build for.
The other two Experiments located at the LHC are ALICE and LHCb.
The ALICE experiment main studies the quark gluon plasma during the runs with lead ion collisions instead of protons.
In this thesis the Scintillating Fibre Tracker (SciFi Tracker) located at the LHCb will be focused at and discussed on the following chapters.

\subsection{The LHCb}

\begin{figure}
  \centering
  \includegraphics[width=0.5\textwidth]{plots/LHCb_facility.jpg}
  \caption{a sideview of the LHCb experiment.}
  \label{fig:LHCb}
\end{figure}

The LHCb experiment\cite{lhcbInfo} is a forward spectrometer covering $2 \less \eta \less 5$ in the pseudorapidity range. This experiments main physics goal is beauty quark physics and for high energies, b- and $\bar{b}$-hadrons are heavily produced in a tight forward direction\footnote{They are also produced in a tight backward direction but the experiment is only build for the forward cone.}. A sideview of the LHCb is shown in figure \ref{fig:LHCb}.
The LHCb consists of several smaller detector components namely the Vertex Locator (VELO) right on the intercation point, two Ring Imaging Cherenkov counter (RICH 1 and RICH 2), in front of the spectrometers lies the Trigger Tracker and behind them the SciFi Tracker which is the important part of this thesis. Further back a Scinitllator Pad Detector (SPD) and a Preshower (PS) are mounted followed by the electromagnetic calorimeter (ECAL) and the hadronic calorimeter (HCAL). In the very back, several muon chambers are mounted for every track that is yet to be determined.

In this section, a general overview about the requirements for the SciFi Tracker as well as the layout will be discribed based on the presentation in the \textit{technical design report}\cite{scifiInfo} of the upgrade.

The upstream and downstream trackers provide a good precision estimate of the momentum of charged particles so that mass resolution of decayed particles can be precisely measured. % this is a sentence i used from the TDR!
For particle identification the reconstructed trajectories of charegd particles are used as input for the RICH detectors.
The limiting factor for the momentum resolution is multiple scattering for tracks with a momentum lower than $\SI{80}{\giga\electronvolt\per\c}$. For tracks with a higher momentum the detector resolution is the limiting factor.

% why is the scifi there
The SciFi Tracker replaced the inner Tracker (IT) and the outer Tracker (OT) and is located in the same place as the downstream trackers that were previously installed.

% data and facts
The instantaneous luminosity after the upgrade is expected to be $\SIrange{1}{2e33}{\per\centi\metre\squared\per\second}$. The bunch spacing will be $\SI{25}{\nano\second}$ and the number of proton-proton interactions per bunch crossing will be $\nu = 3.8$ during the ramp-up phase of the LHC and $\nu = 7.6$ "during the active phase." (how can i write this differently?)

% layout
\subsection{Layout of the Detector}
The SciFi Tracker consists of three (T-)stations T1, T2 and T3 with each having four Layers ($X1, U, V, X2$). The orientation of these planes with respect to the vertical axis are ($\SI{0}{\degree}, \SI{+5}{\degree}, \SI{-5}{\degree}, \SI{0}{\degree}$).
The tilted layers are called stereo layers and serve the purpose of 3D hit localization.
Each layer consists of 8 fibremats
A right-handed coordinate system is used with positive $z$ pointing away from the interaction point following the beam direction. positive $y$ points upwards, towards the surface and positive $x$ and negative $x$ are defined as A-Side and C-Side\cite{scifiInfo}.

% \begin{enumerate}
%   \item layout
%   \item how does it work?
%   \item what else?
% \end{enumerate}

\section{The LHC data cycle}
\label{sec:datacycle}

(not sure if that's a good name, but like, an explanation of how electrical information is turned into hits and then tracks, and also when alignment runs in the system of data taking)

%%% from bachelor thesis %%%

% \section{Die CKM-Matrix\cite{ckm}}
% Die CKM-Matrix, benannt nach Cabibbo, Kobayashi und Maskawa, ist auch als
% Quarkmischungsmatrix bekannt. Sie enth\"allt 3 Winkel und eine komplexe Phase.
% Die Elemente auf der Hauptreihe beschreiben die Zerf\"alle innerhalb der Familie
% und sind demnach nahe an 1. Die \"ubrigen Elemente sind etwas kleiner f\"ur
% Zerf\"alle zwischen benachbarter Familien und sehr stark unterdr\"uckt f\"ur
% Zerf\"alle zwischen der ersten und dritten Familie, wie zum Beispiel das hier
% nicht zu vernachl\"assigende Element $|\symup{V}_{us}|$. \\
% $\left(|\symup{V_{ij}}|\right) = \left[\begin{array}{rrr}
% 0.97446 & 0.22452 & 0.00365 \\
% 0.22438 & 0.97359 & 0.04214 \\
% 0.00896 & 0.04133 & 0.999105 \\
% \end{array}\right] =
% \left[\begin{array}{rrr}
% \symup{V}_{ud} & \symup{V}_{us} & \symup{V}_{ub} \\
% \symup{V}_{cd} & \symup{V}_{cs} & \symup{V}_{cb} \\
% \symup{V}_{td} & \symup{V}_{ts} & \symup{V}_{tb} \\
% \end{array}\right]$
