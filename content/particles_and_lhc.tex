\chapter{Particles and The Large Hadron Collider}
\label{sec:particleslhc}

With good alignment, studies on all Standard Model (SM) particles and hadron states
will improve. Yielding better efficiencies on SM particle measurements will
result in deeper insights regarding physics beyond the Standard Model from high-precision measurements of CP-violating observables as an example.

On a grand scheme upgrading the LHCb and therefore the LHC will bring deeper
insights on Standard Model processes. To understand the universe even better and eventually yield information about the unsolved question of "dark matter" and
"dark energy", what is believed to be the bulk of the universes content.

\section{The Standard Model}
\label{sec:sm}

\begin{figure}
  \centering
  \includegraphics[width=0.75\textwidth]{plots/SM_2018.png}
  \caption{The standard model of particle physics\cite{standard2018}.}
  \label{fig:sm2018}
\end{figure}

The standard model of particle physics\ref{fig:sm2018} describes the known elementary particles and their interactions. It consists of 12 matter particles, the fermions
and five interaction particles. which are called gauge bosons.

The 12 fermions are spin-$\frac{1}{2}$ particles. Six are called leptons and they are sorted into three families, also called flavors (e, $\mu$ and $\tau$) and six are called quarks. Each of those lepton families has a charged lepton\footnote{can have both handnesses} and a left-handed neutrino.
A particle has a left-handed helicity if its spin direction is opposite to the direction of flight. Right-handed helicity particles have a spin direction pointing with the direction of flight.
Neutrinos can only be left-handed since there is no system where the neutrino can be "overtaken" so the momentum switches and therefore the helicity.
% A left-handed chirality isospin doublet and a right-handed chirality singlet can be constructed.
The leptons can couple via the weak-interaction and if they are charged, also via the em-interaction. Neutrinos can only couple via the weak interaction.
Each matter particle also has an antiparticle, with an opposite quantum numbers.

The quarks carry an electric charge as well. In each of the three generations there is one isospin doublet. In the first generation are the two lightest quarks,
up- and down quark, in the second generation the charm- and strange quark and
in the third generation the top- and bottom quark doublet.
Quarks carry a colour charge (anti quarks carry the respective anti-colour), red, green or blue, which is an artificially introduced degree of freedom to guarantee the distinguishability.

%---------- interactions -----------
The interactions are mediated through the gauge bosons.
The three interactions are the electromagnetic(em) interaction,
the weak interaction and the strong interaction. Gravitation does not make a significant contribution.
The gauge boson of the em interaction is the photon which is exchanged between particles.
% here, a feynman diagram would be nice. as well as for every other force
The strength of each interactions is
described by a coupling constant. In the em interaction this is the
fine structure constant\cite{alphas}. The range of the em-interaction is in principle
infinite, but decreases with increasing distance between the interacting particles.
The em interaction is described by quantum electrodynamics.
The potentials are described by operators, which create and annihilate the photons.

The exchange particles of the weak interaction are on the one hand the $W^{\pm}$-bosons and on the other hand the Z-boson.
The weak interaction processes are called currents.
Changing the charge during the interaction by a W-boson is called charged current.
The exchange reaction of a Z boson in, for example, processes such as $e_{\nu} \mu \to e_{\nu} \mu$ is called neutral current.
Analogous to the electromagnetic interaction, the potentials are again understood as
operators, but here there are no propagators. Propagators are
used in \textit{Feynman}-diagrams of QED to represent the interaction particles.
A so-called V-A structure is used here instead. Here, V stands for vectorboson and A is the axialvector.
This structure is needed to disregard the right-handed particles and left-handed
antiparticles, since these lead to the charge-parity violation. Thus the Lorentz factors are adjusted in the following way
\begin{equation*}
  \gamma_{\mu} \to \gamma_{\mu}(1 - \gamma_5)
\end{equation*}
Quarks couple via the strong interaction which is described by the quantum chromodynamic (QCD). The Gauge group of the QCD is $SU\left(N = 3\right)$ where N is the number of introduced colors as a new degree of freedom. The number of generators is therefore $N^2 - 1 = 8$.
The generators are called gluons and they carry color and anticolor, have no mass and carry spin 1.
Gluons can, other than photons, couple to themselves.
Moreover, the coupling constant is $\alpha_s \approx 0.1$.
% The interaction with quarks is described with a potential.
% \begin{align}
%   \symup{V}_{q\bar{q}} &= -\frac{4 \alpha_{s}}{3 r} + \sigma\cdot r
%   \intertext{with}
%   \sigma &= \SI{1}{\giga\electronvolt\per\femto\metre}
% \end{align}
% The common eight gluon-wavefunctions\cite{qcd} are
% \begin{align*}
%   \psi_1 &= |r\bar{g}> & \psi_2 &= |r\bar{b}> \\
%   \psi_3 &= |g\bar{r}> & \psi_4 &= |g\bar{b}> \\
%   \psi_5 &= |b\bar{r}> & \psi_6 &= |b\bar{g}> \\
%   \psi_7 &= \frac{1}{\sqrt{2}}\left(|r\bar{r}> - |g\bar{g}>\right) & \psi_8 &= \frac{1}{\sqrt{6}}\left(|r\bar{r}> + |g\bar{g}> - 2|b\bar{b}>\right) \\
% \end{align*}

The second wavefunction describes a gluon interaction with a blue quark and changing the color to red.

Quarks thus tend to attract each other very strongly. If now
quark and anti quark are moved away from each other, a lot of energy has to be expended. This energy can become so large that new particles can be created.

Due to the Confinement, quarks cannot exist alone. Instead they form bound states, so called hadrons. On the one hand there are the mesons, which consist of a quark
and an antiquark.

These may be from the same family (i.e. [u,d], [c,s], [t,b]), or from
different families. Mesons have a baryon number of 0. Accordingly, quarks carry the baryon number $\frac{1}{3}$. The quarks constructing a meson therefore carries color and the corresponding anticolor.
The second type are baryons. The content consists of either three quarks or
three antiquarks. However, it cannot be that one quark and two antiquarks
and vice versa occur, because baryons must have the baryon number $\symup{B} = 1$. Because baryons are stable final states as well as the mesons, the sum of their quark colors must be white. Therefore, every (anti)color must occur once in a baryon.

\section{Particle decays and hadrons}
\label{sec:decays}


\section{The LHC and LHCb}
\label{sec:lhcandB}

\subsection{The LHC}
The Large Hadron Collider (LHC)\cite{lhcInfo} is the largest particle-accelerator on planet earth. With a circumference of $26,7\si{\kilo\metre}$ it is also the longest ring accelerator and it lies between $45\si{\metre}$ and $170\si{\metre}$ below the surface near Geneva in Switzerland. The tunnel was constructed for the LEP experiment between 1984 and 1989 and is operated by the European Organization for Nuclear Research (CERN). The LHC can produce centre-of-mass energies of $\sqrt{s} = \SI{13}{\tera\electronvolt}$ in proton-proton collisions during Run 2. The design centre-of-mass energy after the upgrade is $\sqrt{s} = \SI{14}{\tera\electronvolt}$.
An image of the accelerators and the experiments is shown in figure \ref{fig:CERN}.

\begin{figure}
  \centering
  \includegraphics[angle=-90, origin=c, width=0.7\textwidth]{plots/CERN_layout.pdf}
  \caption{an overview of the LHC facilities\cite{facilityCERN}.}
  \label{fig:CERN}
\end{figure}

By ionizing hydrogen gas, protons are created and accelerated to $\SI{50}{\mega\electronvolt}$ by the linear accelerator (LINAC 2). Afterwards the beam is injected into the Proton Synchrotron and the Super Proton Synchrotron to a maximum of $\SI{450}{\giga\electronvolt}$ before the beam is injected into the LHC.
The beam consists of several bunches with around $\num{1.15e11}$ protons per bunch and a bunch spacing of $\SI{25}{\nano\second}$.%., which is a collision rate of $\SI{40}{\mega\hertz}$.
The LHC houses four major experiments. ATLAS and CMS are classified as general purpose detectors with a detection range of close to $4\pi$. The interaction in these detectors is located in the very center so that tracks going in every direction can possibly be found. Searches for the Higgs Boson is just one of many physics aspects these detectors are build for.
The other two Experiments located at the LHC are ALICE and LHCb.
The ALICE experiment mainly studies the quark-gluon plasma during the runs with lead ion collisions instead of protons.

\subsection{The LHCb experiment}
\label{sec:upgradeLHCb}

\begin{figure}
  \centering
  \includegraphics[width=0.9\textwidth]{plots/LHCb_facility.png}
  \caption{a sideview of the LHCb experiment\cite{facilityLHCb} after the upgrade.}
  \label{fig:LHCb}
\end{figure}

For high energies, $b$- and $\bar{b}$-hadrons are abundantly produced in forward
direction\ref{fig:bbforward}\footnote{They are also produced in backward direction but the experiment is only build for the forward cone.} with a production cross section of $144 \pm 1 \pm 21 \si{\micro\barn}$ \cite{bbXsection} for $\SI{13}{\tera\electronvolt}$.
The uncertainties are statistical and systematic, respectively.
The LHCb experiment\cite{lhcbInfo} is a forward spectrometer covering $2 \,<\, \eta \,<\, 5$ in the pseudorapidity range. The main physics goal of the LHCb experiment is in the beauty- and charm-quark sector. A sideview of the LHCb is shown in figure \ref{fig:LHCb}.

The Vertex Locator (VELO) is the tracking detector dedicated to primary vertex precision measurements as well as tracking displaced vertices of particles with short lifetimes. The previous VELO used silicon microstrips technology is replaced by 26 tracking layers. The tracking layers use $50\times50\si{\micro\metre}^2$ pixels for a finer hit resolution as well as a better track reconstruction. The previous VELO was $\SI{8.4}{\milli\metre}$ away from the beam pipe. The new VELO will be $\SI{5.1}{\milli\metre}$ away from the beam pipe so that the particles enter the detector earlier and interact with less detector material before the first interaction layer is hit.
This will improve the impact parameter resolution as well as the tracking resolution of the VELO for tracks with low momentum\cite{Piucci}.
For Downstream reconstruction of of particles decaying after the VELO the upstream Tracker (UT) will be utilized. The UT is a necessary component for the improvement of the trigger timing and it also contributes to the improvement of momentum resolution. The UT consists of 4 tracking layers using silicon strips. Analogous to the VELO the inner sensors will be closer to the beam axis in comparison to the current tracker.
Due to the expected radiation dose and particle distribution throughout the tracker, the out region uses a $p^{+}$-in-$n$ technology for the strips and the strips in the middle region uses $n^{+}$-in-$p$ technology to better withstand the higher radiation. The strips closest to the beam use $n^{+}$-in-$p$ technology as well.
The Scintillating Fibre Tracker (SciFi) is located after the magnet region. Therefore measurements regarding particle momentum can be supplied.
Close to the interaction point (IP) the Ring Imaging Cherenkov counter (RICH) are located. Their purpose is $p$, $\pi$ and $K$ particle identification (PID).
The upgraded RICH1 will be able to handle the increasing particle occupancy dude to the increase of the focal length by a factor of $\sqrt{2}$.
The readout rate will be increased from currently $\SI{1}{\mega\hertz}$ to $\SI{40}{\mega\hertz}$, realized by Multianode Photomultipliers (PMTs).
The hadronic calorimeter and electromagnetic calorimeter as well as the four muon chambers are used for $p$, $e$, $\gamma$ and $\mu$ PID. The front-end electronics of the calorimeters are being replaced during the upgrade.

\begin{figure}
  \centering
  \includegraphics{plots/bbar_forward.png}
  \caption{branching ration of $b$ and $\bar{b}$ production in forward direction. Shown is the data versus fixed-order plus next-to-leading logs(FONLL)\cite{forward}\cite{fonll}.}
  \label{fig:bbforward}
\end{figure}

In this section, a general overview about the requirements for the SciFi Tracker as well as the layout will be described based on the presentation in the \textit{technical design report}\cite{scifiInfo} of the upgrade.

The upstream and downstream trackers provide a good precision estimate of the momentum of charged particles so that mass resolutions of decayed particles can be precisely measured.
% this is a sentence i used from the TDR!
For particle identification the reconstructed trajectories of charged particles are used as input for the RICH detectors.
The limiting factor for the momentum resolution is multiple scattering for tracks with a momentum lower than $\num{80}\frac{\text{GeV}}{\text{c}}$. For tracks with a higher momentum the detector resolution is the limiting factor.

The Scintillating Fibre (SciFi) Tracker replaced the inner Tracker (IT) and the outer Tracker (OT)
and is located in the same place as the downstream trackers that were previously installed.

The instantaneous luminosity after the upgrade is expected to be $\SIrange{1}{2e33}{\per\centi\metre\squared\per\second}$ with a bunch spacing of $\SI{25}{\nano\second}$.
The average number of proton-proton interactions per bunch crossing will be between
$\nu = 3.8$ and $\nu = 7.6$.

\subsection{Layout of the SciFi Tracker}

\begin{figure}
  \centering
  \includegraphics{plots/SciFi_Tracker.png}
  \caption{frontal view of the SciFi geometry and readout order\cite{scifiupdate20210311}.}
  \label{fig:scifi}
\end{figure}

The SciFi Tracker consists of three stations T1, T2 and T3 each having four layers ($X1, U, V, X2$). The orientation of these planes with respect to the vertical axis are ($\SI{0}{\degree}, \SI{+5}{\degree}, \SI{-5}{\degree}, \SI{0}{\degree}$).
Since the beampipe is not exactly parallel to the ground the vertical axis is
defined as vertical on the z-axis of the beampipe.
The tilted layers are called stereo layers and serve the purpose of 3D hit localization.
The layers are $\SI{20}{\milli\metre}$ apart from each other in $z$-direction within each station.
Each layer has four quarters with each quarter having five\footnote{six for the last (T-)station.} modules. Each module is constructed from four fibre mats.
A frontal view of the SciFi Tracker is displayed in figure\ref{fig:scifi}.
The global coordinate system used is of right-handed nature with positive $z$ pointing away from the interaction point following the beam direction as seen in figure \ref{fig:LHCb}. positive $y$ points upwards, towards the surface and positive $x$ and negative $x$ are named as A-Side and C-Side respectively\cite{scifiInfo}.
For readout purposes the top and bottom half of each element have inverted x- and y-axis as seen in figure \ref{fig:scifi}.

To ensure an optimal alignment, a well known geometry is key. Therefore, the
fibres within each mat must be aligned within $\SIrange{50}{100}{\micro\metre}$ in $x$-direction and must not be more than $\SI{300}{\micro\metre}$ bent in $z$-direction.
% here a picture of the scifi

\subsection{Scintillating Fibres}
The scintillating fibre material is a polymer with an organic fluorescent dye
added to the polystyrene structure to enhance the yield during the scintillation process.
In order to produce and register a photon signal, the ionization energy is deposited
in the fibre core firstly. The amount of energy need for the polymer to reach an
excited state is just a few electronvolts. The added dye has the particular structure
to match the excitation energy. The dye generates excited energy states when
particles hit the fibre and deposit their energy.
The long fibre mats and the refractive index make sure total reflection on the inside
happens which guides the photons to the silicon photomultipliers (SiPMs).
On the opposite end of the SiPM a full reflective mirror is mounted so the photons travelling to the other end do not get lost but reflected towards the SiPM.
