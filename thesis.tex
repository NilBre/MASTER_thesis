%%%%%%%%%%%%%%%%%%%%%%%%%%%%%%%%%%%%%%%%%%%%%%%%%%%%%%%%%%%%%%%%%%%%%%%%%%%%%%%%
%%%%%%%%%%%%%%%%%%   Vorlage für eine Abschlussarbeit   %%%%%%%%%%%%%%%%%%%%%%%%
%%%%%%%%%%%%%%%%%%%%%%%%%%%%%%%%%%%%%%%%%%%%%%%%%%%%%%%%%%%%%%%%%%%%%%%%%%%%%%%%

% Erstellt von Maximilian Nöthe, <maximilian.noethe@tu-dortmund.de>
% ausgelegt für lualatex und Biblatex mit biber

% Kompilieren mit
% latexmk --lualatex --output-directory=build thesis.tex
% oder einfach mit:
% make

\documentclass[
  tucolor,       % remove for less green,
  BCOR=12mm,     % 12mm binding corrections, adjust to fit your binding
  parskip=half,  % new paragraphs start with half line vertical space
  open=any,      % chapters start on both odd and even pages
  cleardoublepage=plain,  % no header/footer on blank pages
]{tudothesis}


% Warning, if another latex run is needed
\usepackage[aux]{rerunfilecheck}

% just list chapters and sections in the toc, not subsections or smaller
\setcounter{tocdepth}{1}

%------------------------------------------------------------------------------
%------------------------------ Fonts, Unicode, Language ----------------------
%------------------------------------------------------------------------------
\usepackage{fontspec}
\defaultfontfeatures{Ligatures=TeX}  % -- becomes en-dash etc.

% german language
\usepackage{polyglossia}
\setdefaultlanguage{german}

% for english abstract and english titles in the toc
\setotherlanguages{english}

% intelligent quotation marks, language and nesting sensitive
\usepackage[autostyle]{csquotes}

% microtypographical features, makes the text look nicer on the small scale
\usepackage{microtype}

%------------------------------------------------------------------------------
%------------------------ Math Packages and settings --------------------------
%------------------------------------------------------------------------------

\usepackage{amsmath}
\usepackage{amssymb}
\usepackage{mathtools}
\usepackage{bbold}

% Enable Unicode-Math and follow the ISO-Standards for typesetting math
\usepackage[
  math-style=ISO,
  bold-style=ISO,
  sans-style=italic,
  nabla=upright,
  partial=upright,
]{unicode-math}
\setmathfont{Latin Modern Math}

% nice, small fracs for the text with \sfrac{}{}
\usepackage{xfrac}


%------------------------------------------------------------------------------
%---------------------------- Numbers and Units -------------------------------
%------------------------------------------------------------------------------

\usepackage[
  locale=DE,
  separate-uncertainty=true,
  per-mode=symbol-or-fraction,
]{siunitx}
\sisetup{math-micro=\text{µ},text-micro=µ}
% \sisetup{tophrase={{ to }}}
%------------------------------------------------------------------------------
%-------------------------------- tables  -------------------------------------
%------------------------------------------------------------------------------

\usepackage{booktabs}       % \toprule, \midrule, \bottomrule, etc

%------------------------------------------------------------------------------
%-------------------------------- graphics -------------------------------------
%------------------------------------------------------------------------------

\usepackage{graphicx}
%\usepackage{rotating}
\usepackage{grffile}
\usepackage{tikz}
\usepackage{circuitikz}
\usepackage{tikz-feynman}
\usepackage{subcaption}

% allow figures to be placed in the running text by default:
\usepackage{scrhack}
\usepackage{float}
\floatplacement{figure}{htbp}
\floatplacement{table}{htbp}

% keep figures and tables in the section
\usepackage[section, below]{placeins}


%------------------------------------------------------------------------------
%---------------------- customize list environments ---------------------------
%------------------------------------------------------------------------------

\usepackage{enumitem}
\usepackage{listings}
%------------------------------------------------------------------------------
%------------------------------ Bibliographie ---------------------------------
%------------------------------------------------------------------------------

\usepackage[
  backend=biber,   % use modern biber backend
  autolang=hyphen, % load hyphenation rules for if language of bibentry is not
                   % german, has to be loaded with \setotherlanguages
                   % in the references.bib use langid={en} for english sources
]{biblatex}
\addbibresource{references.bib}  % the bib file to use
\DefineBibliographyStrings{german}{andothers = {{et\,al\adddot}}}  % replace u.a. with et al.


% Last packages, do not change order or insert new packages after these ones
\usepackage[pdfusetitle, unicode, linkbordercolor=tugreen]{hyperref}
\usepackage{bookmark}
\usepackage[shortcuts]{extdash}

%------------------------------------------------------------------------------
%-------------------------    Angaben zur Arbeit   ----------------------------
%------------------------------------------------------------------------------

\author{Nils Breer}
\title{Alignment studies for the LHCb SciFi Detector}
\date{2022}
\birthplace{Unna}
\chair{Lehrstuhl für Experimentelle Physik V}
\division{Fakultät Physik}
\thesisclass{Master of Science}
\submissiondate{May 16th 2022}
\firstcorrector{Prof.~Dr.~Albrecht}
\secondcorrector{Prof.~Dr.~Weingarten}

% tu logo on top of the titlepage
\titlehead{\includegraphics[height=1.5cm]{logos/tu-logo.pdf}}

\begin{document}
\frontmatter
\maketitle

% Gutachterseite
\makecorrectorpage

% hier beginnt der Vorspann, nummeriert in römischen Zahlen
\chapter*{Abstract}
\label{sec:abstract}

Due to larger luminosities and higher track multiplicity, detectors with finer granularity are needed.
The LHCb experiment is undergoing a major during the long shutdown of the LHCb between 2019 and 2022. The three tracking detectors downstream of the dipole magnets are undergoing a replacement with a detector that consists of scintillating fibers (SciFi).
The calibration of the new detector to the software regarding
orientation and position is critical for the subsequent performance. This
process is called \textit{alignment} and is part of the trigger and is a crucial part of the real-time analysis of the LHCb.
\\
In this work, the software \textit{alignment} of the SciFi tracker is studied.
So called \textit{misalignment} tests contributed to determine the quality
of the alignment. In addition, tests for the identification of weak modes were performed.
This drew attention to a bias within the SciFi hit clusters which was observed to have an impact on the performance of the alignment procedure.

% more precise and give results!

\chapter*{Kurzfassung}
\label{sec:kurzf}

Aufgrund größerer Luminositäten und höherer Spurmultiplizitäten werden Detektoren mit feinerer Granularität benötigt.
Das LHCb-Experiment wird während der langen Abschaltung des LHCb zwischen 2019 und 2022 einem größeren Umbau unterzogen. Die drei Spurdetektoren hinter den Dipolmagneten werden durch einen Detektor ersetzt, der aus szintillierenden Fasern (SciFi) besteht.
Die Kalibrierung des neuen Detektors mit der Software hinsichtlich
Ausrichtung und Position ist entscheidend für die spätere Leistung. Dieser
Prozess wird \textit{alignment} genannt und ist Teil des Triggers und ein entscheidender Teil der Echtzeitanalyse des LHCb.
\\
In dieser Arbeit wird das Software-\textit{Alignment} des SciFi-Trackers studiert.
Sogenannte \textit{misalignment}-Tests trugen zur Bestimmung der Qualität
des Alignments. Darüber hinaus wurden Tests zur Identifizierung von schwachen Moden durchgeführt.
Dies machte auf einen Bias innerhalb der Cluster aufmerksam welcher einen
nicht-unwichtigen Einfluss auf das Alignment hat.

% \chapter*{Kurzfassung}
% \label{sec:kurzf}
%
% Der LHCb-Detektor erhält ein Upgrade welches in 2019 in Form eines Long
% Shutdowns des LHC begann. Die drei Spurdetektoren hinter den Dipolmagneten wurden
% duch einen Detektor ausgetauscht, welcher szintillierende Fasern verwendet (SciFi).
% Dies ist nur ein Teil des Upgrades. Aufgrund von größeren Luminositäten und der
% steigenden Anzahl an Spurmuliplizitäten werden Detektoren mit feinerer
% Granularität benötigt. Die Kalibrierung des neuen Detektors mit der Software in
% Orientierung und Position ist entscheidend für die spätere Leistung. Dieser
% Vorgang heißt \textit{Alignment}.
% \\
% In dieser Arbeit wird das Software-\textit{Alignment} des SciFi-Trackers studiert.
% Durch Tests verschiedener Parameter Konfigurationen konnte ein gutes
% Alignment erreicht werden. Sogenannte \textit{Misalignment}-Tests trugen dazu
% bei die Qualität des Alignments zu bestimmen. Außerdem wurden Tests zur
% Identifikation von schwachen Moden durchgeführt.
% Dies machte auf einen Bias innerhalb der Cluster aufmerksam welcher einen
% nicht-unwichtigen Einfluß auf das Alignment hat.

\tableofcontents

\mainmatter
% Hier beginnt der Inhalt mit Seite 1 in arabischen Ziffern
\chapter{Introduction}
\label{sec:einleitung}

At the beginning of the $20^{\text{th}}$ century many physicists started research on
elementary particles and the interactions associated with them. The combined
knowledge lead to the construction of most precisely tested theories: the
Standard Model (SM) of particles.
Flavor anomalies show strong tensions with Standard Model and also the recent publication on the W-boson mass calculation is challenging the accuracy of the Standard Model. Despite these phenomena the measurements are in agreement
with the model within the theoretical and experimental uncertainties.
The SM describes every fundamental force except for gravity. There are still open
questions such as the baryon asymmetry of the universe leading to a larger
charge-parity (CP) violation than the SM predicted.
To tackle these problems, high energy experiments such as the LHCb experiment located at the Large Hadron Collider (LHC) at CERN were built for this exact reason.
The LHCb experiment was designed to study beauty and charm quarks with focus on measuring CP-violation and searching for New Physics in rare decays.
To detect these phenomena the threshold for statistical uncertainties has to
be lowered and the amount of data collected needs to be increased. The upgrade
described in section \ref{sec:upgradeLHCb} will increase the instantaneous
luminosity by a factor of five to $\SI{2e33}{\invfb}$ and
the detector readout rate will be at $\SI{40}{\mega\hertz}$. To realize these
hardware and software challenges, the front-end electronics and tracking systems needed upgrades.
To operate the upgraded LHCb experiment at its full potential the software must be
calibrated as well as possible to the physical detector.
\\
\\
The Alignment theory will be described in chapter \ref{sec:alignTheory}.
In chapter \label{sec:story} different sets of constraints, degrees of freedom
and alignable objects called \textit{configuration} will be tested first in order
to study how different configurations influence the alignment. Afterwards several
tests will be performed to analyse the behavior of a misaligned detector and check
if the chosen configuration converges towards an aligned state.

% Since the detector
% cannot be aligned the whole time, it is crucial that misalignments do not stay
% permanent but can be reverted.

The LHC will not run permanently at maximum luminosity. During the restart of the LHC it will run at lower luminosities. Therefore tests are performed to analyse alignment of different luminosity samples. During the alignment studies a bias inside the SciFi hit clustering algorithms was discovered which had an impact on the alignment. The exact changes will be discussed in the final section of chapter \ref{sec:story}.

% theory
\chapter{Particles and The Large Hadron Collider}
\label{sec:particleslhc}

\section{The Standard Modell}
\label{sec:sm}

The standard model of particle physics describes the known elementary particles and their interactions. It consists of 12 matter particles, the fermions
and five interaction particles, which are called vector bosons. The
fermions can be grouped in two categories, six quarks and
six leptons. The quarks as well as the leptons are divided into three generations.
Each matter particle also has an antiparticle, with an opposite charge.
The interactions are obtained from the vector bosons mentioned above.
The three potent interactions are the electromagnetic(em) interaction,
the weak interaction and the strong interaction. Gravitation does not make a significant contribution.
The vector boson of the em interaction is the photon which is exchanged between particles.
% here, a feynman diagram would be nice. as well as for every other force
The strength of one of those interactions is
described by a coupling constant. In the em interaction this is the
fine structure constant\cite{alphas}. The range of the em-interaction is in principle
infinite, but decreases with increasing distance between the interacting particles.
The em interaction is described by quantum electrodynamics.
The potentials are described by operators, which create and annihilate the photons.

The exchange particles of the weak interaction are on the one hand the $W^{\pm}$-bosons and on the other hand the Z-boson.
The weak interaction processes are called currents.
Changing the charge during the interaction by a W-boson is called charged current.
The exchange reaction of a Z boson in, for example, processes such as $e_{\nu} \mu \to e_{\nu} \mu$ is called neutral current.
Analogous to the electromagnetic interaction, the potentials are again understood as
operators, but here there are no propagators. Propagators are
used in \textit{FEYNMAN}-diagrams of QED to represent the interaction particles.
A so-called V-A structure is used here instead. Here, V stands for vectorboson and A is the axialvector.
This structure is needed to disregard the right-handed particles and left-handed
antiparticles, since these lead to the charge-parity violation. Thus one adjusts the Lorentz factors in the following way
\begin{equation*}
  \gamma_{\mu} \to \gamma_{\mu}(1 - \gamma_5)
\end{equation*}
In the strong interaction, the exchange particles are the eight different
gluons. The strong interaction is described by quantum chromodynamics (QCD).
According to this, color is transfered during the interaction. Gluons
have no mass and are spin-1 particles. Gluons carry a color and a different anti-color. Gluons can also couple to themselves. Moreover, the coupling constant $\alpha_s \approx 0.1$. The interaction with quarks is described with a potential.
\begin{align}
  \symup{V}_{q\bar{q}} &= -\frac{4 \alpha_{s}}{3 r} + \sigma\cdot r
  \intertext{mit}
  \sigma &= \SI{1}{\giga\electronvolt\per\femto\metre}
\end{align}
Quarks thus tend to attract each other very strongly. If now
quark and antiquark are moved away from each other, a lot of energy has to be expended. This energy can become so large that new particles can be created.

% now particles
The standard model houses 12 spin-$\frac{1}{2}$ fermions. Six are called leptons and they are sorted into three families, also called falvor (e, μ and τ). Each of those families has a charged lepton\footnote{can have both h\"ndigkeiten} and a left-handed neutrino.
A particle is called left-handed if its spin direction is opposite to the direction of flight.cRight-handed particles have a spin direction pointing with the direction of flight. Therefore, a left-handed isospin doublet and a right-handed singlet is constructed.
The leptons can couple via the weak-interaction and if they are charged, also via the em-interaction. Neutrinos can only couple via the weak interaction.

The quarks are spin-$\frac{1}{2}$-fermions and carry an electric charge as well. In each of the three generations there is one isospin doublet. The quarks are ordered by ascending mass. In the first generation are the two lightest quarks,
up- and down quark, in the second generation the charm- and strange quark and
in the third generation the top- and bottom quark doublet.
Quarks carry a color charge, red, green or blue, which is an artificially introduced degree of freedom to guarantee the distinguishability.
Quarks couple via the strong interaction which is described by the quantum chromodynamic (QCD). The Eichgroup of the QCD is $SU\left(N = 3\right)$ where N is the number of introduced colors. The number of generators is therefore $N^2 -1 = 8$.
The generators are called gluons and they carry "color" and "anticolor".
The common eight gluon-wavefunctions are
\begin{align*}
  \psi_1 &= |r\bar{g}> & \psi_2 &= |r\bar{b}> \\
  \psi_3 &= |g\bar{r}> & \psi_4 &= |g\bar{b}> \\
  \psi_5 &= |b\bar{r}> & \psi_6 &= |b\bar{g}> \\
  \psi_7 &= \frac{1}{\sqrt{2}}\left(|r\bar{r}> - |g\bar{g}>\right) & \psi_8 &= \frac{1}{\sqrt{6}}\left(|r\bar{r}> + |g\bar{g}> - 2|b\bar{b}>\right) \\
\end{align*}
% hier zitieren wo ich das her hab. besser nicht wikipedia!
The second wavefunction describes a gluon interaction with a blue quark and changing the color to red.

Due to the Confinement, quarks cannot exist alone. Instead they form bonding states, so called hadrons. On the one hand there are the mesons, which consist of a quark
and an antiquark.

\begin{equation}
	|\symup{M}\!> = | q\, \bar{q}'\!>
\end{equation}

These may be from the same family (i.e. [u,d], [c,s], [t,b]), or from
different families. Mesons have a baryon number of 0. Accordingly, quarks carry the baryon number $\frac{1}{3}$. The quarks constructing a meson therefore carries color and the corresponding anticolor.
The second type are baryons. The content consists of either three quarks or
three antiquarks. However, it cannot be that one quark and two antiquarks
and vice versa occur, because baryons must have the baryon number $\symup{B} = 1$. Because baryons are stable final states as well as the mesons, the sum of their quark colors must be white. Therfore, every (anti)color must occur once in a baryon.

\begin{align}
	|\symup{B}\!> &= |q q' q''\!> \\
	|\symup{B}\!> &= |\bar{q} \bar{q}' \bar{q}''\!> \,
\end{align}

\section{particle decays and hadrons}
\label{sec:decays}
For you, this is because the bending and momentum of particles (and the location where they decay) is important to the way we can align things. Which sorts of particles can produce long tracks? etc.

\section{The LHC and LHCb}
\label{sec:lhcandB}

\subsection{The LHC}
The Large Hadron Collider (LHC)\cite{lhcInfo} is the most powerfull particle-accelerator on planet earth. With a circumference of $26,7\si{\kilo\metre}$ it is also the longest ring accelerator and it lies between $45\si{\metre}$ and $170\si{\metre}$ below the surface near Geneva in Swizerland. The tunnel was constructed for the LEP experiment between 1984 and 1989 and is operated by the European Organization for Nuclear Research (CERN). The LHC can produce centre of mass energies of $\sqrt{s} = \SI{13}{\tera\electronvolt}$ in proton-proton collisions during Run 2. After the upgrade the LHC will collide particles with the centre of mass energy $\sqrt{s} = \SI{13}{\tera\electronvolt}$.
An image of the accelerators and the experiments is shown in fig. \ref{fig:CERN}\cite{facilityCERN}.

\begin{figure}
  \centering
  \includegraphics[angle=-90, origin=c, width=0.5\textwidth]{plots/CERN_layout.pdf}
  \caption{an overview of the LHC facilities.}
  \label{fig:CERN}
\end{figure}

By ionizing hydrogen gas, protonsare created and accelerated to $\SI{50}{\mega\electronvolt}$ by the linear accelerator (LINAC 2). Afterwards the beam is injected into the Proton Syncrotron and the Super Proton Synchrotron to a maximum of $\SI{450}{\giga\electronvolt}$ before the beam is brought into the LHC.
The beam containts several bunches with around $\num{1.15e11}$ and a bunch spacing of $\SI{25}{\nano\second}$, which is a collision rate of $\SI{40}{\mega\hertz}$.
The LHC houses four major experiments. ATLAS and CMS are classified as general purpose detectors with a detection range of close to $4\pi$. The interaction in these detectors is located in the very center so that tracks going in every direction can possibly be found. Searches for the Higgs Boson is just one of many physics aspects these detectors are build for.
The other two Experiments located at the LHC are ALICE and LHCb.
The ALICE experiment main studies the quark gluon plasma during the runs with lead ion collisions instead of protons.
In this thesis the Scintillating Fibre Tracker (SciFi Tracker) located at the LHCb will be focused at and discussed on the following chapters.

\subsection{The LHCb}

\begin{figure}
  \centering
  \includegraphics[width=0.5\textwidth]{plots/LHCb_facility.jpg}
  \caption{a sideview of the LHCb experiment.}
  \label{fig:LHCb}
\end{figure}

The LHCb experiment\cite{lhcbInfo} is a forward spectrometer covering $2 \less \eta \less 5$ in the pseudorapidity range. This experiments main physics goal is beauty quark physics and for high energies, b- and $\bar{b}$-hadrons are heavily produced in a tight forward direction\footnote{They are also produced in a tight backward direction but the experiment is only build for the forward cone.}. A sideview of the LHCb is shown in figure \ref{fig:LHCb}.
The LHCb consists of several smaller detector components namely the Vertex Locator (VELO) right on the intercation point, two Ring Imaging Cherenkov counter (RICH 1 and RICH 2), in front of the spectrometers lies the Trigger Tracker and behind them the SciFi Tracker which is the important part of this thesis. Further back a Scinitllator Pad Detector (SPD) and a Preshower (PS) are mounted followed by the electromagnetic calorimeter (ECAL) and the hadronic calorimeter (HCAL). In the very back, several muon chambers are mounted for every track that is yet to be determined.

In this section, a general overview about the requirements for the SciFi Tracker as well as the layout will be discribed based on the presentation in the \textit{technical design report}\cite{scifiInfo} of the upgrade.

The upstream and downstream trackers provide a good precision estimate of the momentum of charged particles so that mass resolution of decayed particles can be precisely measured. % this is a sentence i used from the TDR!
For particle identification the reconstructed trajectories of charegd particles are used as input for the RICH detectors.
The limiting factor for the momentum resolution is multiple scattering for tracks with a momentum lower than $\SI{80}{\giga\electronvolt\per\c}$. For tracks with a higher momentum the detector resolution is the limiting factor.

% why is the scifi there
The SciFi Tracker replaced the inner Tracker (IT) and the outer Tracker (OT) and is located in the same place as the downstream trackers that were previously installed.

% data and facts
The instantaneous luminosity after the upgrade is expected to be $\SIrange{1}{2e33}{\per\centi\metre\squared\per\second}$. The bunch spacing will be $\SI{25}{\nano\second}$ and the number of proton-proton interactions per bunch crossing will be $\nu = 3.8$ during the ramp-up phase of the LHC and $\nu = 7.6$ "during the active phase." (how can i write this differently?)

% layout
\subsection{Layout of the SciFi Tracker}
The SciFi Tracker consists of three (T-)stations T1, T2 and T3 with each having four layers ($X1, U, V, X2$). The orientation of these planes with respect to the vertical axis are ($\SI{0}{\degree}, \SI{+5}{\degree}, \SI{-5}{\degree}, \SI{0}{\degree}$).
The tilted layers are called stereo layers and serve the purpose of 3D hit localization.
The layers are $\SI{20}{\milli\metre}$ apart from each other in $z$-direction within each station.
Each layer has four quarters with each quarter having five\footnote{six for the last (T-)station.} modules. Each module is constructed from four fibre mats.
A right-handed coordinate system is used with positive $z$ pointing away from the interaction point following the beam direction. positive $y$ points upwards, towards the surface and positive $x$ and negative $x$ are defined as A-Side and C-Side\cite{scifiInfo}.

To ensure an optimal alignment, a well known geometry is key. Therefore, the fibres must aligned within $\SIrange{50}{100}{\micro\metre}$ in $x$-direction and must not be more than $\SI{300}{\micro\metre}$ bent in $z$-direction.
% here a picture of the scifi

\subsection{Scintillating Fibres}
The scintillating fibre material is a polymer with an organic flourescent dye added to the polystyrene structure to enhance the yield during the scintillation process.
In order to produce and register a photon signal, the ionisation energy is deposited in the fibre core firstly. The amount of energy need for the polymer to reach an excited state is just a few electronvolts.
The added dye has the particular structure to match the excitation energy.
The energy is transferred via the Förster Transfer.
The dye generates excited energy states when particles hit the fibre and deposit their energy. When the states relax, a photon is emitted which then excites other parts of the dye to eventually transfer the energy to the silicon photomultipliers (SiPMs) which are mounted on the outer(?) end of the mats.
The other end of the fibre mats is a full reflective mirror to send the photons towards the SiPMs.

\section{The LHC data cycle}
\label{sec:datacycle}

(not sure if that's a good name, but like, an explanation of how electrical information is turned into hits and then tracks, and also when alignment runs in the system of data taking)

\chapter{The Theory of Alignment}
\label{sec:alignTheory}

short introduction

\section{Track Reconstruction}
\label{sec:kalman}

In order for LHCb to be used for physics, all of the detector hit information has to be converted into tracks, which is a challenging task.
The track reconstruction algorithm needs to find the correct hits from each subdetector to build the track. This can be problematic just because of the amount of tracks per events (roughly 100).
It is crucial to find all particle tracks and also their track parameters which come from the track fit.

A good track fit is needed in order to find to best estimates for the track parameters and covariances. The estimates are used in the event reconstruction to find the correct tracks for each particle and the decay products. The info provided is used in the RICH rings, ECAL and HCAL and muon detectors. With these information, particle and track parameters such as the invariant mass can be measured and vertex origins can be found.
There are several track models that can be used. In general, a track is build from numerous segments which are either straight or curved because of an active magnetic field. Depending on the environment of the track either model is good.
The track segments are called track states and are defined by a position in $x$ and $y$ at a given distance $z$ where the hit was located, then a tanget direction $t_{x,y}$ at the hit position and a momentum parameter acquired from the track curve inside the magnetic field\cite{VanTilburg}.

In order to correctly reconstruct the track it is important to know where the hit is localized and for the upcoming hits, where to particle track came from. From the momentum measurement of the track curvature caused by the magnetic field, the parameter $q/p$ is also added.

\begin{align*}
  \vec{r} = \left(\begin{array}{c} x \\ y \\ t_x \\ t_y \\ \frac{q}{p}\end{array}\right) &\,\, t_x = \frac{\partial x}{\partial z} & t_y = \frac{\partial y}{\partial z}
\end{align*}

The uncertainty of the five-component state vector is a $5\times5$ covariance matrix $C$.
A track state can be anywhere on the trajectory but is easier to choose it at real detection points. Combining the track state with a real measurement point is called \textit{node}.
The propagation from node $k-1$ to node $k$ is described by a propagation function

\begin{equation*}
  \vec{r}_k = f_k(\vec{r}_{k_{-1}}) + \vec{w}_k\,.
\end{equation*}

This means node $k$ is acquired by propagating node $k-1$ through the propagation function $f_k$ and shifting it by the \textit{process noise} $\vec{w}_k$.
% Process noise can be caused by any scattering phenomenon that may have happened.
LHCb uses process noise to model the scattering.
Depending on the type of propagation, linear or curved, a different propagation function is used.
for a linear extrapolation, $f_k$ results in
\begin{equation*}
  f_k \left(\vec{r}_{k-1}\right) = F_k \vec{r}_{k-1}
\end{equation*}
with the transport matrix $F_k$
\begin{gather*}
  F_K = \begin{pmatrix}
    1 & 0 & \Delta z & 0 & 0 \\
    0 & 1 & 0 & \Delta z & 0 \\
    0 & 0 & 1 & 0 & 0 \\
    0 & 0 & 0 & 1 & 0 \\
    0 & 0 & 0 & 0 & 1 \\
  \end{pmatrix}
\end{gather*}
and $\Delta z$ being the difference in z between the nodes
\begin{equation*}
  \Delta z = z_k - z_{k-1}
\end{equation*}

Trajectory information for each node is provided by the real measurement where the relation between measurement $m_k$ and track state at a given node $k$ is defined as

\begin{equation*}
  m_k = h_k(\vec{r}_k) + \epsilon_k
\end{equation*}

with the projection function $h_k$ and \textit{measurement noise} $\epsilon_k$.
So if the detector only measures the $y$ coordinate of state, the projection function
will be
\begin{equation*}
  h_k(\vec{r}_k) = H_k \vec{r}_k
\end{equation*}
with
\begin{gather*}
  H_k = \begin{pmatrix}
    0 & 1 & 0 & 0 & 0 \\
  \end{pmatrix}
\end{gather*}\,.

When measuring more parameters the measurement matrix $H_k$ and projection matrix have dimension $n\times5$ with $n$ being the numbers of parameters measured.

With this track model, $\epsilon_k$ and $w_k$ are random and unknown and have an expectation value of zero.
% They are defined as $W_k \equiv cov(w_k)$ and $V_k \equiv cov(\epsilon_k)$.

% now kalman filter formalism
\section{The Kalman filter method \cite{VanTilburg}}
In general a track is an ensemble of measurements and track states and the Kalman filter method is used to fit tracks.
The idea of the Kalman filter is, to have a starting node and add measurements one by one. In between the addition of measurements, the local track state is updated with the new information.
The Kalman filter method is a $\chi^2$ minimising problem for the measurement of the track. Because of the iterative nature of the method, it is fast und also used in other fields than physics, for example GPS and meteorology.
The three steps of the Kalman filter will be briefly outlined and later discribed in further detail.

The first step is the $\symbf{Prediction}$: The next track state of the trajectory is predicted based on the track state at the previous node.
The second step is the $\symbf{Filter}$ procedure: By using filter equation*s, the prediction is updated with measurement information in this node. The prediction and filter are repeated for each measurement. With more measurements added, the estimate for the best trajectory is the track state after each filter step.
The final step is called $\symbf{Smoother}$: When the trajectory is complete, smoother equation*s are applied from the last node to the previous node. Therefore the information from all measurements is used in both forward- and backpropagation which results in a more
defined track.

\subsection{first Step: Prediction}
For a given state vector at node \textit{k-1}, the prediction for the $k^{\text{th}}$ state vector and its covariance matrix results from the propagation relations

\begin{align*}
  \vec{r}_p^{k-1} &= f_p\left( \vec{r}_{k-1} \right) \\
  \text{Cov}_k^{k-1} &= F_k C_{k-1} F_k^T + Q_k
\end{align*}

The superscript of the statevector shows the amount of information used in the estimate.
That means $\vec{r}_k^n$ is the smoothed state vector which used all information,
$\vec{r}_k^k-1$ is the predicted state vector and $\vec{r}_k^k \equiv \vec{r}_k$ is the filtered state.

$Q_k$ is the process noise in matrix form and it is part of the predicted
covariance matrix $C_k^{k-1}$.
Because the first state cannot take measurements from the previous state, an initial prediction is taken from the track finding algorithm instead.
The predicted residual between the measurement, $m_k$ and the state vector results in
\begin{equation*}
  \text{res}_k^{k-1} = m_k - h_k\left( \vec{r}_k^{k-1} \right)
\end{equation*}
and the corresponding covariance matrix is defined as
\begin{equation*}
  \text{Cov}_{\text{res},k}^{k-1} = V_k + H_k C_k^{k-1} H_k^T\,.
\end{equation*}

Here, $V_k$ is the measurement variance. With these metrics the minimal $\chi^2$ for the optimal track states can be calculated via
\begin{equation*}
  \left( \chi^2 \right)_k^{k-1} =
  \text{res}_k^{k-1} \left(\text{Cov}_{\text{res},k}^{k-1}\right)^{-1} \text{res}_k^{k-1}
\end{equation*}

\subsection{second Step: Filter}
During the filter step, the track state is updated with the measurement information.
Iteratively, each measurement is added and the filtered state $\vec{r}_k$ and the corresponding covariance matrix is calculated via
\begin{align*}
  \vec{r}_k &= \vec{r}_k^{k-1} + G_p \text{res}+k^{k-1} \\
  \text{Cov}_k &= \left(\mathbb{1} - G_k H_k\right) \text{Cov}_k^{k-1}\,,
\end{align*}
where $G_k$ is the gain matrix of dimension $5\times1$ and is defined as
\begin{equation*}
  G_k = C_k^{k-1} H_k^T \left( \text{Cov}_{\text{res},k}^{k-1} \right)^{-1}
\end{equation*}

Afterwards the residuals and its covariance matrix are calculated and the filtered total $\chi^2$ is defined as
\begin{equation*}
  \left( \chi^2_{\text{filter}} \right)_k = \text{res}_k \text{Cov}_{\text{res},k}^{-1} \text{res}_k\,.
\end{equation*}

The prediction and filter procedure is continued for all measurements until the track is fully reconstructed.
Because the last node at $k \, = \, n$ has the most information in it, a backward update is performed to infuse the previous nodes with the same information as in last node.
This is called \textit{smoother}-step.

\subsection{third Step: Smoother}
The smoother function returns the best possible estimate for track states at
the previous nodes. The method used is called \textit{Rauch-Tung-Striebel}-smoother\cite{RTS}.
The idea is to use backward information and construct a smoothed state vector and covariance matrix
\begin{align*}
  \tilde{r}_k^n &= \vec{r}_k + S_k \left( \vec{r}_{k+1}^n - \vec{r}_{k+1}^k \right) \\
  \tilde{C}_k^n &= C_k
\end{align*}
and the Smoother-matrix $S_k$ of dimension $5\times5$
\begin{equation*}
  S_k = C_k F_{k+1}^T \left( C_{k+1}^p \right)^{-1}\,.
\end{equation*}

In order to calculate the smoothed $\chi^2$ the residual and correspending covariance matrix are
\begin{align*}
  \text{res}_k &= m_k - h_k \vec{h}_k^n \\
  \text{Cov}_{\text{res},k}^n &= V_k - H_k C_k^n H_k^T
\end{align*}

The $\chi^2$ is calculated analogously to the one during the filter step with the difference being the new residuals and covariances.

\section{Alignment with Kalman filter track fit}
\label{sec:derivatives}

In principal, minimizing the track residuals is the obvious way to align a detector.
The residual $\vec{\text{res}}_k$ is defined by the difference between a real detector hit and the expected hit position

\begin{equation}
  \text{res}_k = m_k - h_k(\vec{r},\vec{\alpha})
\end{equation}

where $\symbf{h}$ is the measurement model, $\vec{r}$ are the track parameters and $\vec{\alpha}$ are the alignment parameters.
Aligning the SciFi by minimizing the track $\chi^2$ with the same model as used for reconstruction is an advantage. The idea is to use a global covariance matrix in the Kalman filter track fit. Fiiting the tracks will be the form of alignment.
In the following paragraph this form of alignment is briefly described from the in-depth demonstration from Wouter Hulsbergen\cite{HULSBERGEN1}.

Starting with a short revisit about the important aspects of the Kalman track reconstruction. The track $\chi^2$ is defined as

\begin{equation}
  \chi^2 = \vec{\text{res}}^T V^{-1} \vec{\text{res}}\,,
\end{equation}

where $V$ is the track covariance matrix.
The condition for $\chi^2$ to be minimal with respect to a track model $h(x,\alpha)$ with track parameters $x_k$ and alignment parameters $\alpha_k$ are

\begin{equation}
  \frac{\partial\sum_k\chi^2_k}{\partial \alpha} = 0
\end{equation}

 and

 \begin{equation}
   \forall_k \frac{\partial\chi^2_k}{\partial x_k} = 0\,.
 \end{equation}

The subscript $k$ denotes the track not the vector component. For a single track the subscript can be left out. For many tracks to computation is performed in two steps because it is computational too expensive to use a least square expression.
The first step is to estimate track parameters for a starting set of calibration parameters called $\alpha_0$. The second step is to minimize the total $\chi^2$ with respect to $\alpha$ while also taking $x_k$ and $\alpha$ into account.

The total derivative reads
\begin{equation}
  \frac{\symup{d}}{\symup{d}\alpha} = \frac{\partial}{\partial\alpha} +
  \frac{\symup{d}x}{\symup{d}\alpha}\frac{\partial}{\partial x}\,.
\end{equation}

$\frac{\symup{d}x}{\symup{d}\alpha}$ is a derivative matrix and results from the minimal track $\chi^2$ condition and can be expressed by

\begin{equation}
  \frac{\symup{d}}{\symup{d}\alpha}\frac{\partial \chi^2}{\partial x} = 0
\end{equation}

therefore the derivative matrix is definded as
\begin{equation}
  \frac{\symup{d}x}{\symup{d}\alpha} = -\frac{\partial^2\chi^2}{\partial\alpha\partial x} \left( \frac{\partial^2\chi^2}{\partial x^2} \right)^{-1}\,.
\end{equation}

The total $\chi^2$ for a sample of tracks is minimal with respect to $\alpha$ and $x$ can the be described as
\begin{equation}
  \frac{\symup{d}\chi^2}{\symup{d}\alpha} = 0\,.
\end{equation}

For $N$ alignment parameters a system with $N$ coupled non-linear equations is defined.
Linearizing the minimum $\chi^2$ condition around the starting values $\alpha_0$ and solving the linear system for $\Delta\alpha$ yields the solution.
\begin{equation}
  \frac{\symup{d}^2\chi^2}{\symup{d}\alpha^2}\vert_{\alpha_0} \Delta\alpha =
  -\frac{\symup{d}\chi^2}{\symup{d}\alpha}\vert_{\alpha_0}
\end{equation}

Now, with enough constraints inside the alignment the second derivative matrix is invertable and the covariance matrix for $\alpha$ reads

\begin{equation*}
  \text{Cov}(\alpha) = 2 \left( \frac{\symup{d}^2\chi^2}{\symup{d}\alpha^2} \right)^{-1}\,.
\end{equation*}

Higher order derivatives in $\alpha$ are neglected here. The difference in the total $\chi^2$ resulting from a change in $\Delta\alpha$ is given by

\begin{equation*}
  \Delta_{\chi^2} = \frac{1}{2} \left( \frac{\symup{d}\chi^2}{\symup{d}\alpha} \right)^T \Delta\alpha = -\Delta\alpha^T \text{Cov}(\alpha)^{-1}\Delta\alpha
\end{equation*}

The change in total $\chi^2$ is equivalent to the significance of the alignment correction and $\Delta_{\chi^2}$ is used to follow the convergence of an alignment.

% main part
\chapter{Main part}
\label{sec:story}

% I'd like you to think about the work you have been doing on the alignment and what things have been investigated and what conclusions you have made. You want to build the story one piece at a time, so I think it will start with testing some of the constraints from the null tests and going from there. With the type of work you have been doing, you will also have a much larger "future work" and "continuing work" section, I would put it before the conclusion. This will be discussing things like the way we discovered the cluster bias and the way that it is linked to the rotational degrees of freedom, etc. And it can outline the things you know about the way the real detector will be aligned starting very soon.
% \begin{itemize}
%   \item testing some of the constraints from the null tests
%   \item future work: discovered cluster bias + link to rotational degrees of freedom
% \end{itemize}

% taking notes for now so i know what plots to use
% \begin{enumerate}
%   \item started with null tests.
%   \item which constraint does what?
%   \item which degree of freedom moves what part of the scifi?
% \end{enumerate}
\section{general info}

Alignment using tracks and vertices regarding stations, layers and modules. Testing constraints for different degrees of freedom with the goal to find the "perfect" configuration of constraints and alignable degrees of freedoms.
The following pre-installed alignment conditions with the survey constraints being\\
"FT : 0 0 0 0 0 0 : 1 1 1 0.0003 0.0003 0.0003”\\
"FT/T. : 0 0 0 0 0 0 : 1 1 1 0.0003 0.0003 0.0003”\\
"FT/T./Layer(X1|U|V|X2) : 0 0 0 0 0 0 : 0.2 0.2 0.2 0.0001 0.0001 0.0001”\\
"FT/.*Module. : 0 0 0 0 0 0 : 0.1 0.1 0.1 0.001 0.001 0.001"\\
"FT/.*Mat. : 0 0 0 0 0 0 : 0.05 0.05 0.05 0.1 0.1 0.1”\\
were used.

The string is the name of the element, the first set of six numbers are hardcoded parameters for each of the 3 translation degrees of freedom and 3 rotational degrees of freedom (Tx, Ty, Tz, Rx, Ry, Rz) and the second set of six parameters are the corresponding uncertainties.

The scale for the translations are $\si{\milli\metre}$ and the scale for the rotations being $\si{\radian}$. A survey uncertainty of $\num{0.0001}$ stands for $\SI{0.1}{\milli\radian}$.

The alignments have been performed with gaudisplititer.

During the Alignment, lagrange constraints can be utilized to minimize the
$\chi^2$ under the condition
\begin{equation}
  f(\alpha) = 0
\end{equation}
and adding the lagrange parameter $\lambda$ to get
\begin{equation}
  \Delta \chi^2 = \lambda f(\alpha)
\end{equation}\,.

Lagrange constraints are added to fix losely constrained degrees of freedom and can be used for any linear combination of movements.

%include table from https://twiki.cern.ch/twiki/bin/view/LHCb/TAlignmentManual.

\section{Nulltests and software tests}

As a starting point, Alignment v17r1 was used with 5000 events, magnet in upward position and \textit{GoodLongTracks}.
The \textit{GoodLongTracks} have the following cuts and parameters:
\begin{itemize}
  \item minimum $P_{\text{total}} = 5000$ (units? 5000 MeV?)
  \item maximum $P_{\text{total}} = 200 000$ (units? 200 000 MeV = 200 GeV?)
  \item minimum $p_T = 200$ (units?)
  \item maximum $\chi^2 = 5$
  \item track type should be categorized as "long" (what does that mean -> hits in every subdetector)
  \item maximum number of THoles? is 1
\end{itemize}

and for the later used \textit{HighMomentumTTracks} the cuts and parameters are:
\begin{itemize}
  \item minimum $P_{\text{total}} = 50000$ (units?)
  \item track type should be categorized as "Ttrack"
  \item maximum $\chi^2 = 5$
\end{itemize}

% source: https://gitlab.cern.ch/lhcb/Alignment/-/blob/master/Alignment/TAlignment/python/TAlignment/TrackSelections.py

% Results: (sort under the right plots) \\
% \begin{itemize}
%   \item Alignment with Rz (and Rx) takes lots of iterations to converge
%   \item increasing nEvents does not improve it \to weak mode?
% \end{itemize}

At first, a series of tests regarding different degrees of freedom and lagrange constraints is performed to find the optimal solution for the SciFi. This called Nulltests.

The real detector layers are all centered around the beampipe with no shifting in any direction and the goal is to align the layers in the software to mirror the real layers, therefore reduce the shifting as close to zero as possible.

As a baseline the configuration from Florian Reiss is used and improved.
In this part, the steps of testing different configurations is described and analysed.
%What is florians configuration?? -> lxplus, one of the first plots.
%Then describe the configuration i chose.
Figure \ref{dofs} is used to demonstrate which degrees of freedeom can be used
to describe a point in the detector or a shift in coordinates.

Florian made a crucial change in the code in form of a bug fix which resulted
in the following configuration:
\begin{itemize}
  \item dofs: Tx and Rz
  \item constraints: unconstrained
\end{itemize}

\begin{figure}
  \centering
  \includegraphics[width=0.8\textwidth]{plots/june_21/Tx_noRota_allT_1000MU.png}
  \caption{comparison of different configurations without rotational constraints in every station, magnet up and 1000 events. plotted is translation in x versus global z.}
  \label{fig:june_2}
\end{figure}

The measurement points are the mean of each layer, therefore the errorbars on Florians measurement are root-mean-square errors and come from the difference between the C-side and the A-side of the detector layer and is NOT the measurement uncertainty.
The green measurement used the following configuration:
\begin{itemize}
  \item here my config for the green plots -> see lxplus i guess
\end{itemize}
The constraints are selected because... (first of all look them up, then describe)
This measurement only used 1000 events and is only used as a guideline to what the actual distribution could look like.

\begin{figure}
  \centering
  \includegraphics[width=0.8\textwidth]{plots/june_21/Tx_noRota_allT_7000MU.png}
  \caption{comparison of different configurations without rotational constraints in all stations, magnet up and 7000 events. plotted is x translation versus global z.}
  \label{fig:june_2_1}
\end{figure}

In figure \ref{fig:june_2_1} the same measurement is performed for 7000 events to get a better picture. In comparison to \ref{fig:june_2} an overall improvement in Florians configuration is visible and the layer-splitting is reduced but still prominent. The green measurement shows no direct improvement since the layers are already pretty close to zero in x-direction, where we want them.

The following measurements used 3000 events because the performance did not change much when using 5000 or even 7000 events. Also, the time the computer needed to finish an alignment run for 7000 events and 10 iterations was a lot longer than for 3000 events.

maybe show this plot \ref{fig:june_3}
\begin{figure}
  \centering
  \includegraphics[width=0.8\textwidth]{plots/june_21/allT_halfT3_n20_Tx.png}
  \caption{analysed 20 iterations for x translation behavior (look up exact constraints and dofs)}
  \label{fig:june_3}
\end{figure}

\begin{figure}
  \centering
  \includegraphics[width=0.8\textwidth]{plots/june_21/allT_halfT3_Tx_vs_Flo.png}
  \caption{halflayer constraints and full layer constraint, very strict (look up exact constraints and dofs)}
  \label{fig:june_4}
\end{figure}

the figure in \ref{fig:june_4} shows that very strict Tx constraints make Tx look better but when comparing to Tz as we can see in figure \ref{fig:june_5}
\begin{figure}
  \centering
  \includegraphics[width=0.8\textwidth]{plots/june_21/CA_allT_halfT3_Tz.png}
  \caption{compare C-Side to A-Side for Translation in z direction. (look up exact constraints and dofs)}
  \label{fig:june_5}
\end{figure}
a clear layer separation is visible. because of the many constraints that are applied to T3, a compensation is happening in the other two stations.

\begin{figure}
  \centering
  \includegraphics[width=0.8\textwidth]{plots/june_21/CA_allT_halfT3_Tx.png}
  \caption{compare C-Side to A-Side for Translation in x direction. (look up exact constraints and dofs)}
  \label{fig:june_6}
\end{figure}

Looking at figure \ref{fig:june_6}, the last two layers in station 3 are seperation from the first two. Especially the last station should be fixed around zero with the constraints added. The sum of all translations should be zero with each individual layer movement being small.

%\subsection{july plots}

test 3:
\begin{figure}
  \centering
  \begin{subfigure}[b]{0.3\textwidth}
    \centering
    \includegraphics[width=\textwidth]{plots/july_28/Tx.png}
    \caption{Tx versus global z.}
  \end{subfigure}
  \hfill
  \begin{subfigure}[b]{0.3\textwidth}
    \centering
    \includegraphics[width=\textwidth]{plots/july_28/Tz.png}
    \caption{Tz versus global z.}
  \end{subfigure}
  \hfill
  \begin{subfigure}[b]{0.3\textwidth}
    \centering
    \includegraphics[width=\textwidth]{plots/july_28/Rx.png}
    \caption{Rx versus global z.}
  \end{subfigure}
  \caption{Testing a configuration versus florians changes.}
\end{figure}

%\subsection{august plots}
\begin{figure}
  \centering
  \includegraphics[width=0.8\textwidth]{plots/august_13/C_ASide_config5.png}
  \caption{old config 5, plotted C and ASide.}
  \label{fig:aug13_CA_old}
\end{figure}

\begin{figure}
  \centering
  \includegraphics[width=0.8\textwidth]{plots/august_13/CA_side_newconfig5.png}
  \caption{new config 5, plotted C and ASide.}
  \label{fig:aug13_CA_new}
\end{figure}

\begin{figure}
  \centering
  \includegraphics[width=0.8\textwidth]{plots/august_13/2019_c5_old_vs_withRz_MU.png}
  \caption{config5 versus config 5 with Rz constraint for rotational improvement.}
  \label{fig:withRz}
\end{figure}

\begin{figure}
  \centering
  \includegraphics[width=0.8\textwidth]{plots/august_13/combi5_layers_averaged.png}
  \caption{flo with full layer constraint versus config 5}
  \label{fig:floFullL_c5}
\end{figure}

\begin{figure}
  \centering
  \includegraphics[width=0.8\textwidth]{plots/august_13/new_combi5_config.png}
  \caption{constraints for combi 5 used.}
  \label{fig:constraints_c5}
\end{figure}

\begin{figure}
  \centering
  \includegraphics[width=0.8\textwidth]{plots/august_13/newCombi5.png}
  \caption{original new config 5, should be the first good plot to show!!}
  \label{fig:OGconfig5}
\end{figure}

% october plots
update: constraining backlayer and also aligning it (append dofs by : Rx Rz)
Results: \\
\begin{itemize}
  \item improves null tests for Rz but not the only reason why its shifted (cluster bias! expand later)
  \item shearing and scaling constraints do NOT improve this
\end{itemize}

Results of 100mu translation misalignment: \\
100mu translation misalignment matches expected survey uncertainty. BUT layers shifted from 0 !
TODO: FTTrackmonitor to study later splitting in track output

\begin{figure}
  \centering
  \includegraphics[width=0.8\textwidth]{plots/oct_4/TxRz_config5_backlayer.png}
  \caption{dofs Tx Rz and backlayer constraints.}
  \label{fig:oct4}
\end{figure}

\begin{figure}
  \centering
  \includegraphics[width=0.8\textwidth]{plots/oct_6/combi5_added_RZ_backlayer.png}
  \caption{combi 5 with Rz backlayer constraint.}
  \label{fig:oct6}
\end{figure}

% november plots
\section{chi2 tests and weak modes}
In this section, $\chi^2$ are performed in order to study the "goodness" of the alignment since the better the $\chi^2$ after the alignment the better.
The second aspect i want to cover is the impact of potential weak modes also known as "correlated alignment parameters". There are several weak modes that could occur namely \textit{global translation}, \textit{shearing} and \textit{curvature bias}.
weak modes are unaffected by the $\chi^2$ since the residuals do not change.
The effect weak modes have on the alignment are biases regarding track parameters and late convergences.
There are different solutions that can be utilized to reduce the effect from weakmodes such as
\begin{itemize}
  \item $\symbf{using other configurations like magnet off or mass plots for off-axis events}$
  \item $\symbf{utilizing other survey data sets}$
  \item $\symbf{using kinematic and vertex constraints}$
\end{itemize}\,.

\begin{figure}
  \centering
  \includegraphics[width=0.8\textwidth]{plots/nov_19/Figure_2.png}
  \caption{chi2 vs iteration count.}
  \label{fig:fig2}
\end{figure}

\begin{figure}
  \centering
  \includegraphics[width=0.8\textwidth]{plots/nov_21/chi2_vs_iter_all.png}
  \caption{chi2 versus iteration count.}
  \label{fig:chi2iter}
\end{figure}

\begin{figure}
  \centering
  \includegraphics[width=0.8\textwidth]{plots/nov_21/chi2_vs_ntracks_all.png}
  \caption{chi2 versus number of tracks.}
  \label{fig:chi2tracks}
\end{figure}

% december plots
\begin{figure}
  \centering
  \includegraphics[width=0.8\textwidth]{plots/LHCB_week_dec/chi2_vs_iter_normal.png}
  \caption{chi2 versus iteration count normal(?).}
  \label{fig:chi2iterdec}
\end{figure}

\begin{figure}
  \centering
  \includegraphics[width=0.8\textwidth]{plots/LHCB_week_dec/chi2_vs_tracks_normal.png}
  \caption{chi2 versus number of tracks normal.}
  \label{fig:chi2tracksdec}
\end{figure}

% january plats
january 17th plots are for luminosity comparisons.
nu for ramp up luminosity versus lumi during data taking.

\begin{figure}
  \centering
  \includegraphics[width=0.8\textwidth]{plots/jan_17_2022/chi2_iter_low_vs_normal.png}
  \caption{compare different luminosities and plot chi2 versus iteration number as a measurement for weakmodes and alignment.}
  \label{fig:chi2iter_lumi_normal}
\end{figure}

for figure \ref{fig:chi2iter_lumi_normal} redo the plot with correct labels!!

\begin{figure}
  \centering
  \includegraphics[width=0.8\textwidth]{plots/jan_17_2022/chi2_tracks_modulesOnly.png}
  \caption{compare different luminosities and plot chi2 versus number of tracks as a measurement for weakmodes and alignment.}
  \label{fig:chi2tracks_lumi_normal}
\end{figure}

\begin{figure}
  \centering
  \includegraphics[width=0.8\textwidth]{plots/jan_17_2022/tracks_vs_iterations_modulesOnly.png}
  \caption{compare different luminosities and plot number of tracks versus iteration number as a measurement for weakmodes and alignment.}
  \label{fig:chi2iter_lumi_normal}
\end{figure}

january 24th: show that there is a visible difference between low and normal luminosity however the difference is not big enough to differentiate between the two phases during the ramp up and full run. therefore, for the upcoming analysis steps only the "normal/low" luminosity is taken into consideration.

\begin{figure}
  \centering
  \includegraphics[width=0.8\textwidth]{plots/jan_24_2022/compare_with_without_hack.png}
  \caption{because of the clusterbias hack the difference between a measurement with the hack we implemented active and without is shown.}
  \label{fig:cbhack_on_off}
\end{figure}

\begin{figure}
  \centering
  \includegraphics[width=0.8\textwidth]{plots/jan_24_2022/low_normal_with_hack.png}
  \caption{show difference between low and normal luminosity with clusterbias hack active.}
  \label{fig:lumi_low_normal_hack_on}
\end{figure}

% february plots
\begin{figure}
  \centering
  \includegraphics[width=0.8\textwidth]{plots/feb_2_2022/GL_modules_c5_cb_hackactive_low_normal_lumi.png}
  \caption{GoodLong tracks for module alignment and config 5 active. also the clusterbias hack is active comparing low and normal luminosity.}
  \label{fig:GL_lumi_low_normal_hack_on}
\end{figure}

\begin{figure}
  \centering
  \includegraphics[width=0.8\textwidth]{plots/feb_6_2022/100mu_misalignment_samples_compared.png}
  \caption{100mu translation misalignment comparison for different misalignment samples.}
  \label{fig:100muT}
\end{figure}

\begin{figure}
  \centering
  \includegraphics[width=0.8\textwidth]{plots/feb_6_2022/low_lumi_removed_constraints_vs_normal.png}
  \caption{impact of removing constraints from exisiting studies regarding chi2.}
  \label{fig:removeConst}
\end{figure}

% plan on how to sort things
% sources from me:   1. little notebook for first part
% after it was full: 2. sheets with dates

\chapter{Continuing Work}
\label{sec:continue}

\chapter{Future Work}
\label{sec:future}

instead of only doing normal(low) luminosity tests do it for the other luminosity as well (jan 24th)

% conclusion and outlook
\chapter{Conclusion and Outlook}

In order to handle the increasing instantaneous luminosity and read out the data at $\SI{40}{\mega\hertz}$, the LHCb will be upgraded.
The tracking system will be replaced with a single tracker based on scintillating fibres and is currently commissioned. The physics performance is
highly dependent on how well the detector is aligned, since poor alignment leads to systematic biases which can have a negative impact on sensitive asymmetry measurements. It can also lead to worse mass resolution. Therefore it is crucial that the SciFi detector is well aligned.

Starting with the Nulltests, a configuration that can give a good approximation of the real detector and correct for weak modes called "config5 Rz", was found . It was found that constraining $Tx$ and $Tz$ in the last C-frame in station three combined with the rotation around $z$-axis yields the best results in terms of bringing down z-rotation and improving the overall alignment of the SciFi Tracker.
Only constraining station three in the given translational and rotational degrees of freedom improved the alignment for the first two stations regarding $Tz$.

In terms of misalignment studies, it was found that misaligning the Modules in $Rx$, $Ry$ and $Rz$ for every layer in each station by $\SI{0.01}{\milli\radian}$ resulted in misalignment of the SciFi detector especially seen in $Tz$ and $Rz$. The noticable misalignment in $Rz$ is also enlarged by the cluster bias effect.
Misaligning the Modules by $\SI{0.1}{\milli\radian}$ is too large of a misalignment for the SciFi to handle.
On the other hand, misaligning the Modules in $Tx$, $Ty$ and $Tz$ by $\SI{0.001}{\micro\metre}$ or by $\SI{200}{\micro\metre}$ makes no difference and the SciFi cannot fully converge into the state before the misalignments were introduced, but the effects are small enough to be called negligible.

The tests of the different track selections showed a strong correlation between the $\chi^2 / \text{dofs}$ and the number of tracks.
For the GoodLongTracks, the additional alignment of the modules was found to improve the $\chi^2 / \text{dofs}$ as well as the use of two translational and two rotational degrees of freedom compared to only $Tx$ and $Rz$.

Furthermore, a cluster bias was detected which was fixed with a temporary correction. The cluster bias has a negative impact on $Rz$ and increases the rotation around $z$ by a factor of 2. This is currently being monitored and will improve the alignment once a solution is implemented.


% mention continuing work here
With this analysis there are still more open questions. The cluster bias which prevents the alignment from working correctly is currently being analysed and will require more testing.
The alignment using particles is an aspect of the alignment that was not mentioned until this point. What impact particles play during the alignment will therefore be analysed.
Also how particles will help the misalignment of the SciFi detector will be looked at in great detail.
The analysis of weak modes requires much more attention. Knowing the correlation between alignment parameters is therefore a crucial factor in understanding biases in track parameters.


\appendix
% Hier beginnt der Anhang, nummeriert in lateinischen Buchstaben
%\input{content/a_anhang.tex}

\backmatter
\printbibliography
\chapter*{Danksagung}
An dieser Stelle m\"ochte ich mich bei all denen bedanken,
die mir w\"ahrend meiner Bachelorarbeit zur Seite standen und mich immer unterst\"utzt haben.

Zuerst m\"ochte ich mich bei Herrn Professor Dr. Kevin Kr\"oninger
bedanken, durch welchen ich an seinem Lehrstuhl meine Bachelorarbeit
schreiben konnte. Au\ss erdem m\"ochte ich mich bei der Abteilung der
ATLAS Datenanalyse f\"ur die konstruktiven Anregungen bedanken.

Einen gro\ss en Dank spreche ich vor allem meinem Betreuer Dr.
Johannes Erdmann aus, der mich mit voller Unterst\"utzung und
wertvollen Ratschl\"agen und Hilfestellungen durch meine
Bachelorarbeit begleitet hat. Durch ihn habe ich viel
gelernt und bei Fragen konnte er mir stehts weiterhelfen.

Ich m\"ochte mich auch bei Herrn Professor Dr. Bernhard Spaan
f\"ur die Zweitkorrekur meiner Arbeit bedanken.

Mein Dank geb\"urt au\ss erdem Christopher Krause, Jan Lukas
Sp\"ah, Michael Windau, Sebastian L\"utge und Christian Beckmann
f\"ur die fachliche Kompetenz bei Fragen aller Art.

Zuletzt m\"ochte ich meiner Familie und Freunden daf\"ur danken,
dass sie mich w\"ahrend meines gesamten Studiums immer
unterst\"utzt und motiviert haben.

\cleardoublepage
% \thispagestyle{empty}

\includepdf[pages=-]{Eid_nils_breer.pdf}

\end{document}
