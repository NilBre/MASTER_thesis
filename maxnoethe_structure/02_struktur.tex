\chapter{Theoretische Grundlagen}
\begin{enumerate}
    \item \textbf{Einleitung}\\
        In der \emph{kurzen} Einleitung wird die Motivation für die Arbeit
        dargestellt und ein Einblick in die kommenden Kapitel gegeben.
    \item \textbf{Theoretische Grundlagen}\\
        Alles was an theoretischen Grundlagen benötigt wird, sollte auch eher kurz gehalten werden.
        Statt Grundlagenwissen zu präsentieren, eher auf die entsprechenden Lehrbücher verweisen.
        Etwa: Tiefer gehende Informationen zur klassischen Mechanik entnehmen Sie bitte \cite{kuypers}.
    \item \textbf{Ergebnisse} \\
        Der eigentliche Teil der Arbeit, das was getan wurde.
    \item \textbf{Zusammenfassung und Ausblick} \\
        Zusammenfassung der Ergebnisse, Optimierungsmöglichkeiten, mögliche weitergehende Untersuchungen.
\end{enumerate}
In dieser Vorlage ist die Tiefe des Inhaltsverzeichnisses auf \texttt{chapter} und \texttt{section} beschränkt. Möchten Sie diese Beschränkung aufheben, entfernen Sie den Befehl
\begin{verbatim}
            \setcounter{tocdepth}{1}
\end{verbatim}
aus der Präambel oder ändern Sie den Zahlenwert entsprechend. Das Inhaltsverzeichnis sollte für eine Bachelorarbeit auf eine Seite passen.
